\documentclass[twoside]{article}
\usepackage{estilo-ejercicios}
\newcommand{\colapso}{{\searrow\!\!\!\!\searrow}}
%--------------------------------------------------------
\begin{document}

\title{Nilpotent and solvable groups}
\author{Javier Aguilar Martín}
\maketitle

\section{Nilpotent groups}

\begin{ejercicio}{1.1}

\begin{enumerate}

\end{enumerate}

\end{ejercicio}
\begin{solucion}


\end{solucion}

\newpage

\begin{ejercicio}{1.2}

\end{ejercicio}
\begin{solucion}


\end{solucion}




\newpage

\begin{ejercicio}{1.3}

\end{ejercicio}
\begin{solucion}

\end{solucion}

\newpage

\begin{ejercicio}{1.4}

\end{ejercicio}
\begin{solucion}

\end{solucion}

\newpage

\begin{ejercicio}{1.5}

\end{ejercicio}
\begin{solucion}

\end{solucion}

\newpage

\section{Central series}

\begin{ejercicio}{2.1}

\end{ejercicio}
\begin{solucion}

\end{solucion}

\newpage

\begin{ejercicio}{2.2}
 
\end{ejercicio}
\begin{solucion}

\end{solucion}

\newpage

\begin{ejercicio}{2.3}
Probar las siguientes afirmaciones:
\begin{enumerate}
\item $\beta_{(G,X)}(n+m)\leq \beta_{(G,X)}(n)\beta_{(G,X)}(m)$.
\item $w(G,X)=\limsup_{n\to \infty}\sqrt[n]{\beta(G,X)(n)}$ es un límite. 
\item $G$ es de crecimiento exponencial si y solo si $w(G,X)>1$.
\end{enumerate}
\end{ejercicio}
\begin{solucion}

\end{solucion}

\newpage

\begin{ejercicio}{2.4}

\end{ejercicio}
\begin{solucion}

\end{solucion}

\newpage

\section{Invariantes de quasi-isometría: finales}





\end{document}
