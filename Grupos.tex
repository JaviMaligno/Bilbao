\documentclass[twoside, 11pt]{article}
\usepackage{estilo-apuntes}
%\usepackage{amsmath,amssymb}
%\usepackage[utf8]{inputenc}
%\usepackage[spanish]{babel}
%\usepackage{caption}
%\usepackage[]{graphicx}
%\usepackage{enumerate}
%\usepackage{amsthm}
%\usepackage{tikz-cd}
%\usetikzlibrary{babel}
%\usepackage{pgf,tikz}
%\usepackage{mathrsfs}
%\usepackage{bm}  
%\usetikzlibrary{arrows}
%\usetikzlibrary{cd}
%\usepackage[spanish]{babel}
%\usepackage{fancyhdr}
%\usepackage{titlesec}
%\usepackage{floatrow}
%\usepackage{makeidx}
%\usepackage[tocflat]{tocstyle}
%\usetocstyle{standard}
%\usepackage{subfiles}
%\usepackage{color}  
%\usepackage{hyperref}
%\hypersetup{colorlinks=true,citecolor=red, linkcolor=blue}
%\usepackage{eurosym}
%%\usepackage{ntheorem}
%
%
%\renewcommand{\baselinestretch}{1,4}
%\setlength{\oddsidemargin}{0.25in}
%\setlength{\evensidemargin}{0.25in}
%\setlength{\textwidth}{6in}
%\setlength{\topmargin}{0.1in}
%\setlength{\headheight}{0.1in}
%\setlength{\headsep}{0.1in}
%\setlength{\textheight}{8in}
%\setlength{\footskip}{0.75in}
%
%\theoremstyle{definition}
%
%\newtheorem{teorema}{Teorema}[section]
%\newtheorem{defi}[teorema]{Definición}
%\newtheorem{coro}[teorema]{Corolario}
%\newtheorem{lemma}[teorema]{Lema}
%\newtheorem{ej}[teorema]{Ejemplo}
%\newtheorem{ejs}[teorema]{Ejemplos}
%\newtheorem{observacion}[teorema]{Observación}
%\newtheorem{observaciones}[teorema]{Observaciones}
%\newtheorem{prop}[teorema]{Proposición}
%\newtheorem{propi}[teorema]{Propiedades}
%\newtheorem{nota}[teorema]{Nota}
%\newtheorem{notas}[teorema]{Notas}
%\newtheorem*{dem}{Demostración}
%\newtheorem{ejer}[teorema]{Ejercicio}
%\newtheorem{problem}[teorema]{Problema}
%\newtheorem{concl}[teorema]{Conclusión}
%
%\providecommand{\abs}[1]{\lvert#1\rvert}
%\providecommand{\sen}[1]{sen #1}
%\providecommand{\norm}[1]{\lVert#1\rVert}
%\providecommand{\ninf}[1]{\norm{#1}_\infty}
%\providecommand{\numn}[1]{\norm{#1}_1}
%\providecommand{\gabs}[1]{\left|{#1}\right|}
%\newcommand{\bor}[1]{\mathcal{B}(#1)}
%\newcommand{\R}{\mathbb{R}}
%\newcommand{\Z}{\mathbb{Z}}
%\newcommand{\N}{\mathbb{N}}
%\newcommand{\Q}{\mathbb{Q}}
%\newcommand{\C}{\mathbb{C}}
%\newcommand{\Pro}{\mathbb{P}}
%\newcommand{\Tau}{\mathcal{T}}
%\newcommand{\verteq}{\rotatebox{90}{$\,=$}}
%\newcommand{\vertequiv}{\rotatebox{110}{$\,\equiv$}}
%\providecommand{\lrg}{\longrightarrow}
%\providecommand{\func}[2]{\colon{#1}\longrightarrow{#2}}
%\newcommand*{\QED}{\hfill\ensuremath{\blacksquare}}
%\newcommand*\circled[1]{\tikz[baseline=(char.base)]{
%            \node[shape=circle,draw,inner sep=1.5pt] (char) {#1};}}
%\newcommand*{\longhookarrow}{\ensuremath{\lhook\joinrel\relbar\joinrel\rightarrow}}


\begin{document}
%\title{Topología de Superficies}
%\author{Antonio Rafael Quintero Toscano\\ Javier Aguilar Martín}
%\date{Curso 2016/2017}
%\maketitle

\author{Javier Aguilar Martín }
\date{\today}
\title{Nilpotent and solvable groups}

\maketitle


\begin{abstract}

\end{abstract}
PONER EL PREÁMBULO EN INGLÉS PARA QUE LOS ENTORNOS SALGAN EN INGLÉS

	\vfill
	Esta obra está licenciada bajo la Licencia Creative Commons Atribución 3.0 España. Para ver una copia de esta licencia, visite \url{http://creativecommons.org/licenses/by/3.0/es/} o envíe una carta a Creative Commons, PO Box 1866, Mountain View, CA 94042, USA.


\newpage
\tableofcontents

\newpage


\section{Commutators}
Recall that a group $G$ is abelian if and only if $xy=yx$ forall $x,y\in G$, or equivalently, $x^{-1}y^{-1}xy=1$. The left-hand side expression is called \emph{commutator} of $x$ and $y$, denoted $[x,y]$, so $x$ and $y$ commute iff $[x,y]=1$. If we donote $x^y=y^{-1}xy$, $[x,y]=x^{-1}x^y$, or equivalently, $x^y=x[x,y]$. Another immediate consequence of this definition is $xy=yx[x,y]$.

We can see that $abc\cdots xy\cdots wz=y a^yb^y c^y\cdots x^y\cdots wz$, and similarily if we want to move $x$ to the right but with minus sign. $abc=ba^bc=cb^c(a^b)^c=ab^ca^{bc}$. We can do this with commutators, but it's messier.

Recursively we define by $[x_1,\dots, x_i]=[[x_1,\dots, x_{i-1}],x_i]$ (using ``left-norm convention''). 

\subsection{Basic properties of commutators}
\begin{enumerate}
\item $[y,x]=[x,y]^{-1}$
\item $[xy,z]=[xy,z]=[x,z]^y[y,z]=[x,z][x,z,y][y,z]$
\item $[x,yz]=[x,z][x,y]^z=[x,z][x,y][x,y,z]$
\item If $\sigma$ is a homomorphism of groups: $\sigma([x,y])=[\sigma(x),\sigma(y)]$. In particular, $[x,y]^z=[x^z,y^z]$. 
\item $[x^{-1},y]=([x,y]^{x^{-1}})^{-1}$ and $[x,y^{-1}]=([x,y]^{y^{-1}})^{-1}$
\end{enumerate}
Every property is straight forward to prove except the last one, that should be shown by developing $1=[1,y]=[xx^{-1}, y]=[x,y]^{x^{-1}}[x^{-1},y]$

\begin{teorema}[Witt's Identity]
$[x,y^{-1},z]^y[y,z^{-1},x]^z[z,x^{-1},y]^x=1$. 
\end{teorema}
\begin{dem}
Let's check the first factor, and the others will behave similarily. $[x,y^{-1},z]^y=y^{-1}[x,y^{-1},z]y=y^{-1}[x,y^{-1}]^{-1}x^{-1}[x,y^{-1}]zy=y^{-1}[y^{-1},x]z^{-1}[x,y^{-1}]zy=(y^{-1})^x z^{-1}[x,y^{-1}]zy=x^{-1}y^{-1}xz^{-1}x^{-1}yxy^{-1}zy=(xzx^{-1}yx)`{-1}(yxy^{-1}xy)$

When one does the other two, they cancell.
\end{dem}

We can rewrite Witt's formula in the following way:
$[x,y^{-1},z]^y=[[x,y^{-1}],z]^y=[[x,y^{-1}]^y, z^y]$. Recall that $[x,y^{-1}]^y=(([x,y]^{y^{-1}})^{-1})^y=[x,y]^{-1}=[y,x]$. Hence, we have
\[
[y,x,z^y][z,y,x^z][x,z,y^{x}]=1
\]
If we want to start with $x$, then $[x,y,z^x][z,x,y^z][y,z,x^y]=1$.

\begin{defi}
A \emph{metabelian} group is a group $G$ with a normal subgroup $N\trianglelefteq G$ such that both $N$ and $G/N$ are abelian. 
\end{defi}
Abelian groups, cyclic groups and nilpotent groups are examples of metabelian groups. 

\begin{ej}
The dihedral group $D_{2n}=\gene{a,b\mid a^n=b^2=1, a^b=a^{-1}}$. The subgroup $\gene{a}$ is normal by direct computation of conjugates. Then $D_{2n}=\gene{\overline{b}}$ is cyclic, so $D_{2n}$ is metabelian.
\end{ej}

Suppose that we have to classes of groups $\chi$ and $\eta$. A group $G$ is $\chi$-by-$\eta$ if there is a normal subgroup $N$ such that $N\in\chi$ and $G/N\in\eta$. In general, meta-$\chi$=$\chi$-by-$\chi$. 

\begin{ej}
Consider $A_4$ (order 12) and the subgroup $V=\{1, (12)(34), (13)(24),(14)(23)\}$ which is clearly abelian (Klein group). $|A_4/V|=3$ so it is cyclic (prime order implies cyclic). Then $A_4$ is abelian-by-cyclic. The only cyclic normal subgroup of $A_4$ is trivial, so it is not cyclic-by-abelian. 
\end{ej}

Observe that if a group is metabelian, all commutators commute with each other. Indeed, let $c_1, c_2$ two commutators in $G$ metabelian. There exist $N\trianglelefteq G$ with $N$ and $G/N$ abelian. Then, in the quotient, $\overline{c}_1=\overline{c}_2=1$. Then, $c_1,c_2\in N$, and hence $c_1$ and $c_2$ commute. Therefore, in metabelian groups Witt's formula looks nicer:

$$[x,y,z][y,z,x][z,x,y]=1$$

We prove this. $[x,y,z^x]=[[x,y],z[z,x]]=[[x,y], [z,x]][[x,y],z]^{[z,x]}$. Since commutators commute with each other, this is simply $[x,y,z]$. Since commutators commute we can write Witt's identity in the order we prefer. 

\begin{defi}
If $X,Y\subseteq G$ are substes, $[X,Y]=\gene{[x,y]\mid x\in X, y\in Y}$. Note that, $[\{x\},\{y\}]=\gene{[x,y]}$. Recursively define $[X_1,\dots, X_i]$.
\end{defi}
Usually $X$ and $Y$ will be subgroups. 

\begin{defi}
If $G$ is a group and $H,K\leq G$, we say that $K$ \emph{normalizes} $H$ if $K\leq N_G(H)=\{x\in G\mid H^x(=x^{-1}Hx)=H\}=\{x\in G\mid Hx=xH\}$. If that is the case, then $HK=KH$ (which means that $HK$ is a subgroup of $G$). 
\end{defi}
Note that, since $H$ normalizes $K$,  $HK=\bigcup_{k\in K}Hk=\bigcup_{k\in K}kH=KH$, so our claim is true.

\begin{defi}
$K$ \emph{centralizes} $H$ if all elements of $K$ commute with all elements of $H$ (iff any set of generators of $K$ commutes with any set of generators of $H$). This is the same as saying $H$ centralizes $K$.
\end{defi}

\begin{enumerate}
\item $[X,Y]=[Y,X]$
\item $K$ centralizes $H$ iff $[H,K]=1$. In particular $H\leq Z(G)$ ($G$ centralizes $H$) iff $[H,G]=1$. 
\item $K$ normalizes $H$ iff $[H,K]\leq H$. In particular $N\trianglelefteq G$ iff $[N,G]\leq N$. Note that $[h,k]=h^{-1}h^k$, and because $K$ normalizes $H$, $h^k\in H$. 
\item If $\sigma$ is a group homomorphism then $\sigma([H,K])=[\sigma(H),\sigma(K)]$. In particupar $[H,K]^g=[H^g, K^g]$. 
\end{enumerate}
Rememeber that $H\leq G$ is called \emph{characteristic} if $H$ is invariant under all automorphisms of $G$. We write $H char G$. Of course $HcharG\Rightarrow H\trianglelefteq G$. Hence, $H,K char G\Rightarrow [H,K] char G$ and $H,K \trianglelefteq G\Rightarrow [H,K] \trianglelefteq G$. 

\begin{teorema}
$[H,K]\trianglelefteq\gene{H,K}$
\end{teorema}
\begin{dem}
We have to show that $H,K\leq N_G([H,K])$ (they normalize the commutator subgroup). Then, $\gene{H,K}\leq N_G([H,K])$, i.e. $[H,K]\trianglelefteq \gene{H,K}$. It suffices to show that $[h,k]^{h'}\in[H,K]$ forall $h,h'\in H$, $k\in K$. Recall. $[hh',k]=[h,k]^{h'}[h',k]$. Since the first and last term belong to $[H,K]$, then the middle term belongs to $[H,K]$, as we wanted to prove. Similarily we can proceed with the producto on the second coordinate. 
\end{dem}

\begin{teorema}
If $H$ normalizes $L$, then $[HK,L]=[H,L][K,L]$. This happens forall $H$ in particular when $L \trianglelefteq G$, so commutators of normal subgroups are ``linear''. 
\end{teorema}
\begin{dem}
The inclusion to the left is obvious becaus $H,K\subseteq HK$, so $[H,L],[K,L]\subseteq [HK,L]$, and since $[HK,L]$ is a group, we have $[H,L][K,L]\subseteq [HK,L]$. 

For the other inclusion, $[HK,L]=\gene{[hk,l]\mid h\in H,k\in K, l\in L}$. We are going to prove that $[H,L][K,L]$ is a subgroup of $G$ and that each generator $[kh,l]$ of $[HK,L]$ is inside $[H,L][K,L]$. Note that $[hk,l]=[h,l]^k[k,l]=[h,l][h,l,k][k,l]$. On the left, the first term is in $[H,L]$, the last term is in $[K,L]$ and the middle term is in $[L,K]=[K,L]$. Since $H$ normalizes $L$, $[H,L]\leq L$. So the product is in $[H,L][K,L]$. 

Now, recall that $K$ Normalizes $H$ ($K\leq N_G(H)$) implies $HK=KH$ (and hence $HK$ is a subgroup. Let's see that $[H,L]$ normalizes $[K,L]$. We know that $[K,L]\trianglelefteq \gene{K,L}$, so $L$ normalizes $[K,L]$ (every subgroup  of $\gene{K,L}$ does). $H$ normalizes $L$, implies $[H,L]\leq L$, and we're done. 
\end{dem}

\begin{teorema}[Philip Hall's three subgroup lemma]
If $N\trianglelefteq G$ and $H,K,L\leq G$, then IF $[H,K,L],[K,L,H]\leq N$, then also $[L,H,K]\leq N$. 
\end{teorema}
\begin{dem}
We may assume $N=1$ (passing to the quotient). Let us first prove that $[l,h,k]=1$ for all $l\in L, h\in H, k\in K$. Take $x=h^{-1}, y=k,z=l$ in Witt's identity. The first term in the identity belongs to $[H,K,L]=1$ and the second to $[K,J,H]=1$. We then have $[l,h,k]^{h^{-1}}=1$, hence $[l,h,k]=1$. Let us see thtat $[L,H,K]=1$ By definition, $[L,H,K]=[[L,H],K]$, where $[L,H]$ centralizes $K$ and viceversa. The generators of $[L,H]$ are of the form $[l,h]$, so $[[l,h],k]=1$. 
\end{dem}

\section{Central series}
\begin{defi}[lower central series]
It is a series $\{\gamma_i(G)\}_{i\geq 1}$ where $\gamma_1(G)=G$ and $\gamma_i(G)=[\gamma_{i-1}(G),G]$ for $i\geq 2$. 
\end{defi}
These groups are characteristic, so they are normal in $G$. Normality implies that $\gamma_i(G)=[\gamma_{i-1}(G),G]\subseteq\gamma_{i-1}(G)$, so they form a descending series. 

\begin{teorema}
$[\gamma_i(G),\gamma_j(G)]\leq \gamma_{i+j}(G)$ for all $i,j\geq 1$. 
\end{teorema}
\begin{dem}
Induction on $i$. $[\gamma_i(G),\gamma_j(G)]=[\gamma_{i-1}, G, \gamma_j(G)]\overset{?}{\leq} \gamma_{i+j}(G)\trianglelefteq G$. Take $\gamma_{i+j}(G)$ as $N$ in Hall's 3 subgroups lemma. It is enought to show that $[\gamma_{i-1}, G, \gamma_j(G)]\leq \gamma_{i+j}(G)$ and $ [\gamma_j(G),\gamma_{i-1}(G),G)]\leq\gamma_{i+j}(G)$. For $i=1$ it is true. $[G,\gamma_j(G),\gamma_{i-1}(G)]=[\gamma_{j+1}(G),\gamma_{i-1}(G)]\leq \gamma_{i+j}(G)$ (true for $i-1)$. Similarily $[\gamma_j(G),\gamma_{i-1}(G),G]\leq \gamma_{j+1-1}(G),G]=\gamma_{i+j}(G)$.  
\end{dem}

\begin{teorema}
$\gamma_i(G)=\gene{[g_1,\dots, g_i]\mid g_j\in G}$
\end{teorema}
\begin{dem}
Inclusion to the left is obvious. To the right, induction on $i$. The statement is true for $i=2$ in general. Call $N=\gene{[g_1,\dots, g_i]\mid g_j\in G}$. We show that $N$ is normal in $G$. For all $x\in G$, $N^x=\gene{[g_1,\dots, g_i]^x\mid g_j\in G}=\gene{[g_1^x,\dots, g_j^x]\mid g_j\in G}$. Since conjugation is an automorphism, we get all the elements of $G$, and therefore $N\trianglelefteq G$. Then we can assume $N=1$ and prove $\gamma_i(G)=1$. By definition, $\gamma_i(G)=[\gamma_{i-1}(G),G]$, so $\gamma_i(G)=1$ iff $\gamma_{i-1}(G)\leq Z(G)$ iff all generators of $\gamma_{i-1}(G)$ commute with all elements of $G$. By induction, the generators can be taken of the form $[g_1,\dots, g_{i-1}]$. So this equivalent to say that $[g_1,\dots, g_i]=1$ for all $g_i\in G$. 
\end{dem}

Observe that $G'=[G,G]=\gamma_2(G)$. Take into account that $G'$ is characterised by following property: $G'$ is the smallest normal subgroup of $G$ giving an abelian quotient ($G/G'$ is the largest abelian quotient of $G$, and it's called its \emph{abelianisation}). Given $N\trianglelefteq G$, $G/N$ is abelian iff $[\overline{x},\overline{y}]=\overline{1}$ $\forall \overline{x},\overline{y}\in G/N$ iff $[x,y]\in N$ $\forall x,y\in G$ iff $G'\leq N$. 

\begin{teorema}
Let $G=\gene{X}$ be a group. Then $\gamma_i(G)=\gene{[x_1,\dots, x_i]^g\mid x_j\in X, g\in G}=\gene{[x_1,\dots, x_i]\mid x_j\in X}^G.$
\end{teorema}

In general, if $H\leq G$ is not necessarily normal in $G$, that's because some conjugates are missing. The \emph{normal closure} of $H$ in $G$ is the smalles normal subgroup of $G$ containing $H$: $H^G=\gene{h^g\mid h\in H,g\in G}=\gene{H^g\mid g\in G}$. 

\begin{dem}
Call $N=\gene{[x_1,\dots, x_i]^g\mid x_j\in X, g\in G}\trianglelefteq G$. The inclusion to the left is obvious, so let's prove the other one. We may assume that $N=1$. Is then $\gamma_i(G)=1$? By definition $\gamma_i(G)=[\gamma_{i-1}(G),G]$. Is $\gamma_{i-1}(G)\leq Z(G)$? Let us understand that when $i=1$, the commutator is just an element, so using this base case we do induction on $i$. Assume that $\gamma_{i-1}(G)=\gene{[x_1,\dots, x_{i-1}]^g\mid x_j\in X,g\in G}$. It is enough to show that all generators commute with the elements of $G$. Since $Z(G)$ is a normal subgroup, it is enought to show it for the commutators $[x_1,\dots, x_{i-1}]\in Z(G)$, which is the same as saying that $[x_1,\dots, x_{i-1}]$ commutes with every $x_i\in X$ (since $G=\gene{X}$). But $[[x_1,\dots, x_{i-1}],x_i]=[x_1,\dots, x_i]\in N=1$. 



\end{dem}

Since $[x_1,\dots, x_i]^g=[x_1,\dots, x_i][x_1,\dots, x_i,g]$ and the second factor belongs to $\gamma_{i+1}(G)$, we have the following corollary

\begin{coro}
If $G=\gene{X}$, then $\gamma_i(G)=\gene{[x_1,\dots, x_i],\gamma_{i+1}(G)\mid x_j\in X}$, or what is the same $\gamma_i(G)/\gamma_{i+1}(G)=\gene{[\overline{x}_1,\dots, \overline{x}_i]\mid x_j\in X}$
\end{coro}

Note that in a central series, each quotient $\gamma_i(G)/\gamma_{i+1}(G)$ is abelian since $[\gamma_i(G),\gamma_i(G)]\subseteq [\gamma_i(G),G]=\gamma_{i+1}(G)$. If $G$ is finitely generated, then $\gamma_i(G)$ need not be finitely generated but the quotient by $\gamma_{i+1}(G)$ is by the last corollary. And therefore we can apply the structure theorem to this quotient, but to know something about the precise stricture we need the number of generators and the orders of the generators. We know that $d\left(\gamma_i(G)/\gamma_{i+1}(G)\right)\leq d^i$.

When $X=G$ it is not necessary to take normal closure in the theorem, but in general it is, though we can refine the condition over $X$.

\begin{coro}
Let $G=\gene{X}$, where $X$ is a \emph{normal} subset of $G$, meaning that it is invariant under conjugation by elements of $G$. Then $\gamma_i(G)=\gene{[x_1,\dots, x_i]\mid x_j\in X}$.
\end{coro}

\begin{ej}
$D_{2n}=\gene{a,b\mid a^n=b^2=1, a^b=a^{-1}}$. In this case $X=\{a,b\}$. Since $a^b=a^{-1}\notin X$, $X$ is not normal, so we cannot apply the corollary. Let's apply the theorem. Since $[a,b]=a^{-1}a^b=a^{-2}$, $D'_{2n}=\gene{[a,b]}^{D_{2n}}=\gene{a^2}^{D_{2n}}=\gene{a^2}$ (since $\gene{a}\trianglelefteq D_{2n}$ the same holds for $\gene{a^2}$).
\end{ej}

\begin{ej}
$G'=\gene{[x,y]\mid x,y\in G}$. Suppose that we want to compute $G''=(G')'$. By definition it is $\gene{[u,v]\mid u,v\in G'}$. The set of generators of the form $[x,y]$ with $x,y\in G$ is normal in $G$, so also in $G'$. Then, $G''=\gene{[[x,y],[z,t]]\mid z,y,z,t\in G}$. One can easily iterate this method to derive generators of higher derived groups.
\end{ej}

\begin{ej}
We are going to construct a group $G=B\ltimes A$ ($B$ acts on $A$ via automorphisms), where $B=\gene{b}\cong C_\infty$ and $A=\gene{a_1}\times\cdots\times\gene{a_n}\cong C_\infty\times\cdots\times C_\infty$. The action automorphism defining the action of $b$ on $A$ is given by $a_i\mapsto a_ia_{i+1}$ for $i=1,\dots, n_1$ and $a_n\mapsto a_n$. This means that $a_1^b=a_1a_2,\dots, a_{n-1}^b=a_{n-1}a_n$ and $a_n^b=a_n$, so $[a_1,b]=a_2,\dots,[a_{n-1},b]=a_n$ and $[a_n,b]=1$. Clearly $G=\gene{A,B}=\gene{b,a_1,\dots, a_{n-1},a_n}=\gene{b,a_1}$ since all $a_i$ can be generated by $a_1$ and $b$ by conjugation. Is $G'=\gene{[a,b]}=\gene{a_2}$? No, because $a_3,\dots, a_n\in G'$ but they're not in $\gene{a_2}$. The problem is that $\gene{a_2}$ is not normal in $G$.

If we take $b,a_1,\dots, a_n$ as generators, $\gene{[a_1, b],\dots, [a_{n-1},b]}^G=\gene{a_2,\dots, a_n}^G=G'$ (note that $[a_i,a_j]=1$ since they lie in an abelian group). But now $\gene{a_2,\dots, a_n}\trianglelefteq G$, which can be proven directly. NO VEO QUE ESTO SEA VERDAD, ¿$a_i^{a_1}$ ESTÁ?
\end{ej}

If a group $G$ is finitely generated, let us write $d(G)$ for the smallest number of generators of $G$. If $G$ is abelian and $H\leq G$, by the structre theorem, $d(H)\leq d(G)$. This is not necessarily true if $G$ is not abelian. For instance, in the previous example, $d(G)=2$ and $d(A)=n$. One can modify this example to show that a group $H$ can even fail to be finitely generated. 

%\begin{teorema}
%If $G/G'=\gene{\overline{X}}$ and $\gamma_{i-1}(G)/\gamma_i(G)=\gene{\overline{Y}}$, then $$\gamma_{i-1}(G)/\gamma_i(G)=\gene{\overline{[x,y]}\mid x\in X,y\in Y}.$$
%\end{teorema}
%
%\begin{dem}
%EXERCISE
%\end{dem}
%
\begin{ej}
If $G=\gene{X}$ and $|X|=d$, then also $G/G'=\gene{\overline{X}}$. Hence $\gamma_2(G)/\gamma_3(G)=\gene{\overline{[x,y]}x,y\in X}$. There are $\binom{d}{2}$ generators. If $d=2$, say $G=\gene{a,b}$, there is just 1, namely $[\overline{a},\overline{b}]$. -
\end{ej}

\begin{defi}
Let $G$ be a group in which there is a bound for the orthers of all the elements (in particular, all elements are of finite order). Then if $e=\mathrm{lcm}(o(g)\mid g\in G)$, which exists because there is a bound, then $g^e=1$ forall $g\in G$. We call $e$ the \emph{exponent} of $G$ and write $\exp(G)$ to denote it.
\end{defi}

Note that this exponent is not the maximum order. For example, in $S_3$ the maximum order is 3 but the exponent is 6.

\begin{ej}
If $A=\gene{a_1,\dots, a_d}$ is abelian with all generators of finite order. Then an arbitrary element of $a$ is $a=a_1^{i_1}\cdots a_d^{i_d}$. For $e=\mathrm{lcm}(o(a_i))$ then $a^e= 1$. In this particular case the exponent of $A$ is the least common multiple of the generators of $A$. It is radically false if the group is not abelian, as we will see in the next example.
\end{ej}

\begin{ej}
$D_\infty=\gene{a,b\mid b^2=1, a^b=a^{-1}}=\gene{b}\ltimes \gene{a}\cong C_2\ltimes C_\infty$ where te action is given by $a^b=a^{-1}$. This group can be generated by $\{ba,b\}$, which have both order 2, but there are elements of infinite order in the group. 
\end{ej}

\begin{lemma}
Let $x\in\gamma_i(G)$ and $y\in\gamma_j(G)$. Recall that $[x,y]\in[\gamma_i(G)),\gamma_j(G)]\subseteq\gamma_{i+j}(G)$. Write $o(x\mod N)$ for the order of $\overline{x}$ in $G/N$. Then $o([x,y]\mod \gamma_{i+j+1}(G))$ divides $\gcd(o(x\mod\gamma_{i+1}(G)), o(y\mod\gamma_{j+1}(G)))$. 
\end{lemma}
\begin{proof}
Let us show that $o([x,y]\mod \gamma_{i+j+1}(G))$ divides $o(x\mod \gamma_{i+1}(G))$. Call the last number $e$, so $x^e\in\gamma_{i+1}(G)$ and $[x,y]^e\in\gamma_{i+j+1}(G)$. We may assume $\gamma_{i+j+1}(G)=1$. If $x$ commutes with $[x,y]$, then $[x,y]=[x^e,y]$, so let's prove that. $[[x,y],x]=\in\gamma_{j+i+1}(G)=1$ so they commute. Now $[x^e, y]\in\gamma_{i+j+1}(G)$ and we're done.
\end{proof}

\begin{teorema}
$\exp(\gamma_i(G)/\gamma_{i+1}(G))$ divides $\exp(\gamma_{i-1}(G)/\gamma_i(G))$.
\end{teorema}
\begin{dem}
$\gamma_i(G)/\gamma_{i+1}(G)=\gene{[\overline{x},\overline{y}]\mid x\in\gamma_{i-1}(G),y\in G}$ and it is abelian, so $\exp(\gamma_i(G)/\gamma_{i+1}(G))=\mathrm{lcm}(o([\overline{x},\overline{y}])$, which by thelemma divides $o(x\mod\gamma_i(G))$. By definition, this dives $\exp(\gamma_{i-1}(G)/\gamma_i(G))$. 
\end{dem}

\begin{coro}
If $G/G'$ is finite, then $\gamma_i(G)/\gamma_{i+1}G)$ is finite for all $i\geq 1$. 
\end{coro}
\begin{dem}
If $G/G'$ is finite, $\exp(G/G')$ is well defined (finite) and in addition $G/G'$ is finitely generated. Then $\gamma_i(G)/\gamma_{i+1}(G)$ is finitely generated and abelian, so it is finite, and of course $\exp(\gamma_{i-1}(G)/\gamma_i(G))$ is finite. 
\end{dem}
\begin{ej}
EN EL PAPEL
\end{ej}

\begin{defi}
A seires of subgroups $N_{r+1}\leq N_r\leq\cdots\leq N_1$ of a group $G$ is \emph{central} if $[N_i,G]\leq N_{i+1}$ for $i=1,\dots, r$. The number $r$ is called the \emph{length} os series.
\end{defi}

Why \emph{central}? The condition $[N_i,G]\leq N_{i+1}$ is equivalent to say that iff the quotient $\overline{G}=G/N_{i+1}$ is defined, then $[\overline{N}_i,\overline{G}]=\overline{1}$ iff $\overline{N}_i\leq Z(\overline{G})$, i.e $N_i/N_{i+1}\leq Z(G/N_{i+1})$. Observe that in fact $N_i$ must be normal, because $[N_i,G]\leq N_{i+1}\leq N_i]$. 

The most obvious example of central series is the lower central series. If we refine a central series, it is again a central series: if we introduce $N_{i+1}\leq K\leq N_i$, then $[N_i,G]\leq N_{i+1}\leq K$ and $[K,G]\leq [N_i,G]\leq N_{i+1}$. One thing we can do to produce a central series from a normal subgroup $N$, we can start with $N_1=N$ and then $N_2=[N,G]$, $N_3=[N,G,G]$,etc. (This is precisely the lower central series when $N=G$). Another way is going up from the condition $N_i/N_{i+1}\leq Z(G/N_{i+1})$, taking the case of the equality, which means $N=N_{r+1}$, $N_r=Z(G/N_{i+1})$ and so on.

\begin{defi}
The \emph{upper central series} is the central series obtained in the last way in the previous paragraph starting with $N=1$: $Z_0(G)=1$, $Z_i/Z_{i-1}(G)=Z(G/Z_{i-1}(G))$. In particular, $Z_1(G)=Z(G)$ and $Z_2(G)/Z(G)=Z(G/Z(G))$. 
\end{defi}

$Z_i(G)(G/Z(G))=Z_i(G)/Z(G)\Rightarrow Z_i(G)(G/Z_j(G)=Z_{i+j}(G)/Z_j(G)$.

\begin{teorema}
The following conditions are equivalent:
\begin{enumerate}
\item There is a central series from 1 to $G$.
\item The lower central series (LCS) ends in $1$.
\item The upper central series (UPS) ends in $G$.
\end{enumerate}
\end{teorema}

\begin{dem}
That 1 is equivalent to 2 and 3 is trivial, so we're going to proof 1 implies 2. Let $1=N_{r+1}\leq N_r\leq \cdots \leq N_1=G$ by a central series by 1. $N_1=G$, $\gamma_2(G)=[N_1,G]\leq N_2$, $\gamma_3=[\gamma_2(G),G]\leq [N_2,G]\leq N_3$, and so on until $\gamma_{r+1}(G)\leq N_{r+1}=1$. $N_{r+1}=Z_0(G)=1$. $N_r/N_{r+1}\leq Z(G/N_{r+1}))Z(G)/N_{r+1}\Rightarrow N_r\leq Z(G)$. $N_{r-1}/N_r\leq Z(G/N_r)\Rightarrow N_{r-1}/Z(G)\leq  Z(G/Z(G))=Z_2(G)/Z(G)\Rightarrow N{r-1}\leq Z_2(G)$. If we continuo, we teach $G=N_1\leq Z_r(G)\Rightarrow Z_r(G)=G$. 
\end{dem}

Note that this proof implies that $\gamma_{i}(G)\leq N_i\leq Z_{r+1-i}$ (whence the names lower and upper). Furthermore, the length of the LCS and UCS of $G$ is at most $r$ (the length of the original central series).

Because of the theorem, if $G$ is nilpotent of nilpotency $c$, comparing the LCS and UCS $\gamma_{c-i+1}(G)\leq Z_i(G)$ for $i=0,\dots, c$.

Using the proof of $[\gamma_i(G),\gamma_j(G)]\leq \gamma_{i+j}(G)$ (using Hall's 3 subgroup lemma in the same way), we can more generally prove that if $N_{r+1}\leq\cdots\leq N_1$ is a central series, then  $[N_i,\gamma_j(G)]\leq N_{i+j}$. In particular, if you apply this to the upper central series $Z_i(G)$, $[Z_i(G), \gamma_j(G)]\leq Z_{i-j}(G)$ (remember that in the UCS the index goes up). As a special case, $[Z_i(G),\gamma_i(G)]=1$ (they commute elementwise). If $i-j<0$, understand this subgroup as 1. This generalises the fact that $[Z(G),G]=1$ because $Z(G)=Z_1(G)$ and $G=\gamma_1(G)$. 

There exists a relationship among exponents in the UCS in a similar way than the case of LCS, namely, $\exp(Z_{i+1}(G)/Z_i(G))$ divides $\exp(Z_i(G)/Z_{i-1}(G)$, but there is no control on the number of generators of one of the quotients in terms of the number of de generators of the previous quoetients. 

For every primer number $p$ and any $n\in\N$, there exists a finite $p$-group satisfying: $1\leq Z(G)\leq Z_2(G)=G$, with the order of $Z(G)$ being $p$ (so $\exp Z(G)=p$) and the order of $G$ being $p^{2n}$ ($\exp Z_2(G)/Z(G)=p$, which means that this group consists of $2n$ copies of $C_p$, so it needs $2n$ generators). This example can be modified to make $|Z_2(G):Z(G)|=\infty$. 

\section{Nilpotent groups}

\begin{defi}
A \emph{nilpotent} group is one for which the equivalent conditions of the previous theorem hold. In that case, by the observation above, the length of the LCS and UCS is the same, and it is less than or equal to the length of any other central series. This number $c$ is called the \emph{nilpotency class} of $G$ and is determined by any of the condition $\gamma_{c+1}(G)=1$ or $Z_c(G)=G$. 
\end{defi}


Then, we say that $G$ is nilpotent of class $c$ if $\gamma_{c+1}(G)=1$ (and not earlier) or equivalently if $Z_c(G)=G$ (and not earlier). Note that abelian groups are exactly nilpotent groups of class 1, because $Z(G)=G$. At the same time, $G$ is abelian iff $[x_1,x_2]=1$ for any pair of generators of $G$. Since for $G=\gene{X}$ $\gamma_i(G)=\gene{[g_1,\dots, g_i]\mid g_j\in G}=\gene{[x_1,\dots, x_i]^g\mid g\in G, x_j\in X}$, $G$ is nilpotent of class at most $c$ iff $[g_1,\dots, g_{c+1}]=1$ $\forall g_j\in G$ iff $[x_1,\dots, x_{c+1}]=1$ $\forall x_j\in X$.

\begin{ej}
If $p$ is a prime, then all finite $p$-groups are nilpotents.

\begin{lemma}
If $G$ is a finite $p$-group and $1\neq N\trianglelefteq G$, then $N\cap Z(G)\neq 1$. In particular, if $G\neq 1$, then $Z(G)\neq 1$. 
\end{lemma}
\begin{proof}
Consider the partition of $G$ into conjugacy classes. The length of a conjugacy class $x^G=\{x^g\mid g\in G\}$ is given by the formula $|x^G|=|G:C_G(X)|$, which divides $|G|$ (a power of $p$). Therefore $|x^G|$ is a power of $p$, and then it has only one element or $p | |x^G|$. In the first case we have $x^g=x$ for every $g\in G$, so $x\in Z(G)$. Note that $N$ is a union of conjugacy classes. Are there non-trivial elements whose conjugacy class consists only in one element? Write $N$ as union of the conjugacy classes that have one element (those in the center) and the ones whose order is a power of $p$. Then $|N|=pm+|N\cap Z(G)|$. Hence $p$ divides $|N\cap Z(G)|$ and thus $N\cap Z(G)\neq 1$. 

Alternatively, we can assume that there is only one element in the center and then $|N|=pm+1$, which is not divisible by $p$ (contradiction).
\end{proof}

Why is a finite $p$-group nilpotent? If $G=1$ it is trivial. If $G\neq 1$ and $Z(G)=G$ we'r done, otherwise $G/Z(G)\neq 1$. $Z_2(G)/Z(G)=Z(G/Z(G))\neq 1$, so $Z(G)<Z_2(G)$. Sice the group if finite and each step produces a larger group, it ends in $G$ at some point.
\end{ej} 

\begin{ej}
The direct product of finitely many nilpotent groups is nilpotent. If $G=G_1\times\cdots G_n$, then $\gamma_i(G)=\gamma_i(G_1)\times\cdots\times \gamma_i(G_n)$. Actually, the nilpotency class of $G$ is the maximum of the nilpotency classes of the factors.
\end{ej}

\begin{ej}
A direct product of finitely many finite $p$-groups (for possibly different primes $p$) is nilpotent. We will prove that the converse holds: $G$ finite nilpotent implies $G$ is a direct product of finite $p$-groups. 
\end{ej}

\begin{ej}
$G=B\ltimes A=\gene{b}\ltimes \gene{a_1}\times\cdots\times\gene{a_n}\cong C_\infty\ltimes C_\infty\times\cdots\times C_\infty$. The action is given by $a_i^b=a_ia_{i+1}$ ($i=1,\dots, n-1$), $a_n^b=a_n$. Then $\gamma_i(G)=\gene{a_i,\dots, a_n}$ ($\gamma_{n+1}(G)=1$). Then $G$ is nilpotent of class $n$, so there are groups of any nilpotency class. 

There are also finite examples (taken from finite $p$-groups). Let $p$ a prime and $G$ having the same expresion by now $\gene{a_i}=C_p$. We shouldn't choose $\gene{b}=C_p$, because the order of the automorphism induce by the action of $b$ must divide the order of $b$. In this case, it can be checked that the automorphism described before is a power of $p$ depending on $n$, say $p^{\alpha(n)}$. Hence, we take $\gene{b}\cong C_{p^{\alpha(n)}}$. The calculation of the LCS is the same, so we have found a finite $p$-group of class $n$.  
\end{ej}

\begin{ej}
Semidirect poduct do not preserve nilpotency: $D_{2n}=\gene{b}\ltimes\gene{a}$ is only nilpotent if $n$ is a power of 2, as we have seen in the exercises. As a consequence, nilpotency is not preserved under extensions: $1\to N\to G\to G/N\to 1$ with $N$ and $G/N$ nilpotent doesn't imply $G$ is nilpotent. But we will see that if $N\trianglelefteq G$ with $N$ nilpotent and there is a group $N'$ in the middle of the central series of $N$ such that $G/N'$ is nilpotent, then $G$ is nilpotent.
\end{ej}

\begin{ej}
A subgroup of a nilpotent group is nilpotent. Let $G$ be a group and $H,K$ nilpotent subgroups. It is not true in general that $\gene{H,K}$ is nilpotent (take the previous example with $H=\gene{a}$ and $K=\gene{b}$). This not true even when $HK$ es a subgroup.
\end{ej}

\begin{teorema}
The product of two (and so, of finitely many) nilpotent normal subgroups of a group is nilpotent. 
\end{teorema}
\begin{dem}
Let $L,N\trianglelefteq G$ be nilpotent. We prove that $LN$ is nilpotent (when the groups are normal the product is a normal subgroup). Say $\gamma_{r+1}(L)=1=\gamma_{s+1}(N)$, let us see that $\gamma_{r+s+1}(LN)=1$. $$\gamma_{r+s+1}(LN)=[LN,\dots, LN]=\prod_{K_i\in\{L,N\}}[K_1,\dots, K_{r+s+1}]$$

Since there are $r+s+1$ subgroups $K_i$ which can only be $L$ or $N$, one of these two possibilities happens:
\begin{enumerate}
\item $L$ appears at least $r+1$ times
\item $N$ appears at least $s+1$ times.
\end{enumerate}

For the first, the commutetor is a subgroup of $[L,\dots, L]$ (as many times as there were) because $[L,N]\leq L$ by normality. Since $[L,\dots, L]\leq \gamma_{r+1}(L)=1$ we are done since the other case is analogue.

\end{dem}


\begin{nota}
The theorem is not necessarily true for inifinitely many factors (see exercises). What subgroup do we mean with inifinitely many factors?
\begin{enumerate}
\item Forget about ``product'' and take the subgroup they generate.
\item Interpret the product of infinitely many factors as consisting all finite products form by taking elements from the factors. 
\end{enumerate}

Actually both interpretations are equivalent. 
\end{nota}

Now suppose $G$ is finite and consider the product of all nilpotent normal subgroups of $G$ (there are finitely many). This product is nilpotent, but not only this, but it is the largest nilpotent normal subgroup of $G$ by definition. We call it the \emph{fitting subgroup} of $G$ and write it as $F(G)$. We have as a consequence of this definition that $G$ is nilpotent iff $F(G)=G$. It can be checked that $F(G)$ is characteristic in $G$. Recall that a non-abelian simple group $G$ cannot be nilpotent, because the are only two normal subgroups, so the only way to reach 1 is that $G$ is abelian. Thus, if $G$ is non-abelian simple group, then $F(G)=1$. 

\begin{ej}
In $S_3$ there is only one normal subgroup (there are a total of 4 maximal subgroups, generated by 2-cycles), the one generated by $\gene{(123)}$, so we can form $S_3\to \gene{(123)}\to 1$ and $F(S_3)=\gene{(123)}$. 
\end{ej}


\begin{teorema}[Properties of nilpotent groups] Let $G$ be a nilpotent group. Then:
\begin{enumerate}
\item $1\neq N\trianglelefteq G\Rightarrow  [N,G]<N$. In particular, $G'<G$ if $G\neq 1$. 
\item $1\neq N\trianglelefteq G\Rightarrow N\cap Z(G)=1$. In particular, if $G\neq 1$, then $Z(G)\neq 1$. 
\item Every subgroup $H$ of $G$ is subnormal (there exists a sieres $H=H_0\trianglelefteq\cdots\trianglelefteq H_r= G$).
\item $H<G\Rightarrow H<N_G(H)$ (the normalizer condition). 
\item Every maximal subgroup of $G$ is normal in $G$. As a consequence, the index in $G$ is a prime number.
\end{enumerate}
\end{teorema}

Note that if the index of $H$ in $G$ is a prime number, the subgroup is maximal, since any group in the middle of $G$ and $H$ must have index 1. In the case of $G=A_4$, the subgroup $H=\gene{(123)}$ (order 3) is of index 4, but $A_4$ has no subgroups of order 6, so it is maximal. 

\begin{dem}\
\begin{enumerate}
\item If by contradiction $[N,G]=N$ then $1\neq N=[N,G]=[N,G,G]=\cdots=[N,G,\dots, G]\leq \gamma_{c+1}(G)=1$ ($c$ times $G$). 
\item Since $1\neq N$ and $N>[N,G]$, suposse $[N,G]\neq 1$. Then $N>[N,G]>[N,G,G]>\cdots>[N,G,\dots,G]=1$ ($c$ times). There is some point in which $[N,G,\dots, G]\neq 1$  ($r$ times) but $[N,G,\dots, G,G]=1$. Then the first of these commutators is a subgroup of $Z(G)\cap N$, implying that $N\cap Z(G)\neq 1$.
\item We multiply the LCS by $H$: $H=H\gamma_{c+1}(G)\leq\cdots\leq H\gamma_2(G)\leq G=H\gamma_1(G)$. Is $H\gamma_{i+1}(G)\trianglelefteq H\gamma_i(G)$? Equivalently, does $H\gamma_i(G)$ normalize $H\gamma_{i+1}(G)$? It is enough to show that $H$ and $\gamma_i(G)$ normalize $H\gamma_{i+1}(G)$. $H\leq H\gamma_{i+1}(G)$ normalizes $H\gamma_{i+1}(G)$. What about $\gamma_i(G)$? $[H\gamma_{i+1}(G),\gamma_i(G)]\leq [G,\gamma_i(G)]=\gamma_{i+1}(G)\leq H\gamma_{i+1}(G)$. 

\item Consider a series as in 3: $H=H_0\trianglelefteq H_1\trianglelefteq\cdots\trianglelefteq H_r=G$ (we may assume that $H_i\neq H_j$ for $i\neq j$). Since $H_0<H_1$ and it is normal, then $H_1\leq N_G(H)$.  
\item $M$ maximal in $G$ implies $M<G$, which by the normalizer condition implies $M<N_G(M)\leq G$. Since $M$ maximal in $G$ we have indeed $N_G(M)=G$, hence $M\trianglelefteq G$. 

A general fact: if $M$ is a maximal subgroup of $G$ and $M$ is normal, then $|G:M|$ is a prime. Since $M$ is normal and maximal, the only subgroups of $G/M$ are the trivial one and $G/M$, because of the correspondency theorem between subgroups of the quotient and intermediate subgroups between $G$ in $M$. Then $G/M$ has prime order, and this is why $|G:M|$ is a prime. 
\end{enumerate}
\end{dem}

These properties characterise nilpotent groups if we restrict to finite groups with one exception: there are finite non-nilpotent groups which satisfy 2. 

\begin{ej}
Take $G=SL_2(K)=\left\{\begin{pmatrix}
a & b\\
c & d
\end{pmatrix}=A\in M_2(K)\mid \det(A)=1\right\}$. Then all normal subgroups of $G$ are either $G$ or contained in $Z(G)=\left\{\begin{pmatrix}
\lambda & 0\\
0 &\lambda
\end{pmatrix}\mid \lambda^2=1\in K\right\}$. If $K$ is a finite field (so $G$ is finite) and $|K|$ is odd, $K^*$ is a cyclic group of order $|K|-1$ (this number is even, so 2 divides it) so there exists $\lambda\in K$ of order 2. $Z(G)\neq 1$ in this case. But then $1\neq N\trianglelefteq G$ implies that either $N=G\Rightarrow N\cap Z(G)=Z(G)\neq 1$ either $N\subseteq Z(G)$. But $G$ is not nilpotent: $G/Z(G)=PSL_2(K)$ is a non-abelian simple group, implying that $G/Z(G)$ is not nilpotent, so $G$ is not nilpotent.
\end{ej}

\begin{teorema}[Characterization of finite nilpotent groups]
If $G$ is finite, then any of the properties 1,3,4 and 5 imply that $G$ is nilpotent, so they are equivalent to nilpotency in finite groups, as are the following two
\begin{enumerate}
\item[6] All Sylow subgroups are normal.
\item[7] $G$ is the direct product of its Sylow subgroups.
\end{enumerate}
\end{teorema}
\begin{dem}
We have seen the implication $1\Rightarrow$ $G$ nilpotent.  We are going to prove $G$ nilpotent implies  1 and 3, and then $3\Rightarrow 4\Rightarrow 5\Rightarrow 6\Rightarrow 7\Rightarrow G$ nilpotent.

\begin{itemize}
\item If $G=1$, then it is nilpotent. If $G\neq 1$, then $G>[G,G]$. If $[G,G]=1$, $G$ is nilpotent. Otherwise, iterate $[G,G]>[G,G,G]$ and so on. Since $G$ is finite, this series stabilises, so there is some $\gamma_i(G)=1$.

\item (All maximal subgroups are normal implies all Syllow subgroups are normal) Say $|G|=p_1^{n_1}\cdots p_t^{n_t}$ with $p_i$ different primes. Choose a Sylow $p_i$-subgroup $P_i$ for every $i=1,\dots, t$. Is $G=P_1\times\cdots\times P_t$? ($G=P_1\cdots P_t$ and for every $i$ $P_i\trianglelefteq G$ and $P_i\cap P_1\cdots P_{i-1}P_{i+1}\cdots P_t=1$). For every $i_1,\dots, i_s\in \{1,\dots, t\}$ we have $|P_{i_1}\cdots P_{i_2}|=p_{i_1}^{n_{i_1}}\cdots p_{i_2}^{n_{i_s}}$ using the formula $|HK|=\frac{|H||K|}{|H\cap K|}$ when $H$ and $K$ are subgroups. Since $P_i$ are normal, their product is a subgroup so we can iterate the formula in order to get the cardinality of the whole product. Note that $P_{i_1}\cdots P_{i_{s-1}}\cap P_{i_s}$ is 1 because the two intersecting groups have relatively prime orders.  

\item (All Sylow subgroups are normal implies $G$ is the direct product of its Sylow groups) We need two lemmas for this proof.
\begin{lemma}(Frattini argument)
Let $G$ be a finite group and let $N\trianglelefteq G$. If $P$ is a Sylow subgroup of $N$, then $G=N_G(P)N$. 
\end{lemma}
\begin{proof}
We proof only inclusion to the right, because the other one is obvious. Let $g\in G$. We have $P^g\subseteq N^g=N$ and $|P^g|=|P|$. Then $P^g$ is also a Sylow subgroup of $N$ for the same prime. Two Sylow subgroups of a group are conjugate in that group, so $P$ and $P^g$ are conjugate in $N$. This means that $P^{gn}=P$ for some $n\in N$. Then $gn\in N_G(P)$, so we can write $g=gn\cdot n^{-1}$ as an element of $N_G(P)$ times an element of $N$. 
\end{proof}

\begin{lemma}
Let $G$ be a finite group and $P$ a Sylow subgroup of $G$. Then if $N_G(P)\leq H$, we have $N_G(H)=H$.
\end{lemma}
\begin{proof}
Apply Frattini Argument to  $N_G(H)$, $H$ and $P$, since $P$ is a Sylow subgroup of $H$. Then $N_G(H)=N_{N_G(H)}(P)H$. It is clear that $N_{N_G(H)}(P)\subseteq N_G(P)$ (in fact they are equal, but we don't care) and $N_G(P)\subseteq H$, so $N_G(H)=H$. 
\end{proof}

Let us prove the implication. By contradiction, suppose $P$ is a non-normal Sylow subgroup. Then there exists $M$ maximal in $G$ such that $N_G(P)\leq M<G$.  But then $N_G(M)=M$, but then $M\not\trianglelefteq G$\footnote{\url{https://proofwiki.org/wiki/Normal_Subgroup_iff_Normalizer_is_Group}} gives a contradiction. 
\end{itemize}
\end{dem}

Let us see more basic properties of finite $p$-groups. 

\begin{teorema}
Let $G$ be a finite $p$-group. Then 
\begin{enumerate}
\item If $H\leq G$, then there exists a series $H_0=1\leq H_1\leq\cdots \leq H\leq \cdots \leq H_n=G$ where $|H_{i+1}:H_i|=p$ and $H_i\trianglelefteq H_{i+1}$. 
\item If $N\trianglelefteq G$ then there exists a series $N_0=1\leq N_1\leq\cdots \leq N\leq \cdots \leq N_n=G$ where $|N_{i+1}:N_i|=p$ and $N_i\trianglelefteq G$.
\end{enumerate}
\end{teorema}
\begin{dem}\
\begin{enumerate}
\item We use 2. Since $G$ is nilpotent, $H$ is subnormal. We have $H\trianglelefteq G$. We can fill this by applying two to get the indices equals $p$. 
\item By induction on $|G$. If $|G|=1$, trivial. Assume $|G|>1$. If $N\neq 1$ then $N\cap Z(G)\neq 1$. Choose a subgroup (an element) $N_1$ of order $p$ inside $N\cap Z(G)$ (exists because all subgroups of all subgroups of a $p$-groups divide the order, and we can form an element of order $p$ from an element of order $p^k$; one can also use Cauchy theorem that says that for every prime divisor of the order there exists an element of that order).  Then we have $1=N_0\leq N_1$. Go to $G/N_1$, which contains $N/N_1$ as a normal subgroup. By induction there is a series as we want
\[
1=N_1/N_1\leq N_2/N_1\leq \cdots N/N_1\leq\cdots\leq G/N_1
\]
Use the correspondence theorem to get a series in $G$. If $N=1$, then choose any non-trivial $K\trianglelefteq G$ and use the series given by $K$. 
\end{enumerate}

\end{dem}


\begin{coro}
A finite nilpotent group $G$ has (normal) subgroups of order $d$ for every divisor $d$ of $|G|$. 
\end{coro}

How can we know how many elements are needed to generate a finite $p$-group (the smallest possible number of generators $d(G)$?

\begin{defi}
If $G\neq 1$ is a finite group, the \emph{Frattini subgroup} of $G$ is $\Phi(G)=\bigcap\{$ maximal subgroups of $G\}$. It is customary to take $\Phi(G)=1$ if $G=1$. 
\end{defi}

Note that it is necessary for $G$ to be finite, since there are groups like $\Q$ that do not have maximal subgroups, although the Frattini subgroup can be redefined to include infinite groups, since for every element $q$, there are maximal subgroups among the subgroups that do not containg that element. 

\begin{teorema}
Let $G$ be a finite group and let $X\subseteq G$. If $G=\gene{X,\Phi(G)}$, then also $G=\gene{X}$. For this reason, the elements of $\Phi(G)$ are called \emph{non-generators} of $G$. A consequence of this is that if $H\leq G$ and $G=H\Phi(G)$, then $G=H$. 
\end{teorema}
\begin{dem}
By contradiction, suppose $\gene{X}<G$. Since $G$ is finite, we can find a maximal subgroup $M$ such that $\gene{X}\leq M$. But also $\Phi(G)\leq M$, so $\gene{X,\Phi(G)}\leq M<G=\gene{X,\Phi(G)}$, a contradiction.
\end{dem}

\begin{defi}
An \emph{elementary abelian $p$-group}, where $p$ is a prime, is an abelian group in which $x^p=1$ for all elements (all non-identity elements are of orther $p$).
\end{defi}

If an elementary abelian $p$-group is finite then it is isomorphic to $C_p\times\cdots\times C_p$. 

\begin{nota}
An elementary abelian $p$-group $G$ can be seen as a vector space over the field $\F_p=\Z/\Z_p$. The product by scalar is defined by $\overline{\lambda}g=g^{\lambda}$ for $g\in G$ and $\overline{\lambda}\in\F_p$. Since the order of $g$ is $p$, this operation is well-defined. 
\end{nota}

\begin{teorema}[Burnside Basis Theorem]
Let $G$ be a finite $p$-group. Then
\begin{enumerate}
\item $G/\Phi(G)$ is an elementary abelian $p$-group, and so it can be regarded as an $\F_p$-vector space. 
\item $\{x_1,\dots, x_d\}$ is a minimal set of generators of $G$ iff $\{x_1\Phi(G),\dots, x_d\Phi(G)\}$ is basis of $G/\Phi(G)$. 
\item We can determine $d(G)$ by the condition $|G:\Phi(G)|=p^{d(G)}$. 
\end{enumerate}
\end{teorema}
\begin{dem}\
\begin{enumerate}
\item Let $M_1,\dots,M_k$ be all maximal subgroups of $G$. Then $\Phi(G)=M_1\cap\cdots\cap M_k$. Not that in a nilpotent group every maximal subgroup is normal and has prime index. In this case, since $G$ is a $p$-group, each $M_i$ has index $p$. Then $G/M_i\cong C_p$. Consider the canonical homomorphism $\theta:G\to G/M_1\times\cdots\times G/M_k\cong C_p\times\cdots\times C_p$. The image of $\theta$ is a subgroup and it is not trivial, so $\theta$ is surjective. In addition $\ker\theta=M_1\cap\cdots\cap M_k=\Phi(G)$. Then $G/\phi(G)\cong C_p\times\cdots\times C_p$ and then it is elementary abelian. 

\item Immediate consequence of $G=\gene{X}\Leftrightarrow G=\gene{X,\Phi(G)}\Leftrightarrow G/\Phi(G)=\gene{\overline{X}}$. 

\item By 2 $d(G)=\dim_{\F_p}G/\Phi(G)$. If $\overline{x}_1,\dots, \overline{x}_{d(G)}$ is a basis of $G/\Phi(G)$, $G/\Phi(G)=\{\lambda_1 \overline{x}_1+\cdots \lambda_{d(G)}\overline{x}_{d(G)}\mid \lambda_i\in\F_p$ all different$\}$. The amount possible linear combinations are $|G/\Phi(G)|=p^{d(G)}$ because there are $p$ options for each coeficient. 
\end{enumerate}
\end{dem}

\begin{coro}
In a finite $p$-groups all minimal sets of generators have the same cardinality, namely $d(G)$. 
\end{coro}

Recall that this is false even for cyclic groups. One ccan even show that, for a given natural number $N$, one can find a finite cyclic group $G$ (so $d(G)=1)$ where there is a minimal set of generators with $n$ elements. 

\begin{teorema}
Let $G$ be a finite $p$-group. Then
\begin{enumerate}
\item $\Phi(G)$ is the smallest normal subgroup of $G$ giving elementary abelian quotient (giving quotient isomorphic to $C_p\times\cdots\times C_p$).
\item $\Phi(G)=G'\cdot G^p$, where $G^p=\gene{g^p\mid g\in G}$. 
\end{enumerate}
\end{teorema}
 \begin{dem}\
 \begin{enumerate}
 \item We prove that $G/N$ is elementary abelian iff $\Phi(G)\leq N$. Firs, implication to the left. Let us see that $G/\Phi(G)$ is elementary abelian. $G/N\cong \frac{G/\Phi(G)}{N/\Phi(G)}$, so also $G/N$ is elementary abelian. 
 
 Now to the right. If $G/N$ is elementary abelian, $G/N$ can be seen as an $\F_p$-vector space and maximal subgroups are the same as maximal subspaces. The intersection of all maximal subspaces of a vector space is $\{0\}$. This tells us that $\Phi(G)\leq N$ via correspondence theorem, because we get this inclusion by taking only the maximal subgroups containing $N$, and hence we get it intersecting even more. 
 
 \item $G/N$ elementary abelian iff $[\overline{x},\overline{y}]=\overline{1}$ $\overline{x}^p=\overline{1}$ $\forall x,y\in G$ iff $[x,y]\in N$, $x^p\in N$ forall $x,y\in G$ iff $G'GP\subseteq N$. Now $G'G^p$ satisfying the characterizing property of $\Phi(G)$ we showed in 1, so $\Phi(G)=G'G^p$. 
 \end{enumerate}
 \end{dem}
 
 Note that $G'G^p$ can be obtained in 2 steps:
 \begin{enumerate}
 \item Calculate $G'$ (relatively easy if we know generators).
 \item Go to the abelianisation $G/G'$ and take $p$-th powers there (instead of in $G)$. This is easy if we know generators: $G/G'=\gene{\overline{x}_1,\dots, \overline{x}_n}$ implies $(G/G')^p=\gene{\overline{x}^p_1,\dots, \overline{x}_d^p}$ (this can be absolutely wrong ig the group is not abelian). But now $(G/G')^p=G^pG'/G'$, so by the correspondence theorem $G'G^p=\gene{x_1^p,\dots, x_d^p}G'$. 
 \end{enumerate}
 
 Once we know $\Phi(G)$, we can calculate all maximals subgroups of $G$ more easily (especially in the case of finite $p$-groups). We can calculate of maximal subspaces on the quotient and by the correspondence we get of maximal subgroups containing $\Phi(G)$, but by definition every maximal subgroup contains $\Phi(G)$, so we get all.
 

\begin{ej}
Suppose that $G$ is a finite $p$-group with $d(G)=2$. Then $G/\Phi(G)\cong C_p\times C_p$ (dimension 2). Its maximal subspaces are those generated by a non-zero vector. There are $p^2-1$ non-zero vectors and there are $p-1$ generators of every such subspace, so there are $p+1$ subspaces of dimension 1 (maximal subspaces). 
\end{ej}

\begin{teorema}
Let $G$ be a finite group. Then if $N\trianglelefteq G$ and contains $\Phi(G)$: $N/\Phi(G)$ nilpotent implies $N$ nilpotent. As a consequence:
\begin{enumerate}
\item $\Phi(G)$ is nilpotent, and so $\Phi(G)\leq F(G)$.
\item $G/\Phi(G)$ nilpotent implies $G$ nilpotent. 
\end{enumerate}
\end{teorema}
Proving that a group $G$ is nilpotent, one can then star assuming wlog that $\Phi(G)$ going to the quotient by 2. 

\begin{dem}
Let us prove that every Sylow subgroup $P$ of $N$ is normal in $N$. $P\Phi(G)/\Phi(G)$ is a Sylow subgroup of $N/\Phi(G)$, that is nilpotent. Then $P\Phi(G)/\Phi(G)\trianglelefteq N/\Phi(G)$. Every Sylow group is conjugate to another one, so being normal implies that it is the only Sylow group and from Sylow theory we know that $P\Phi(G)/\Phi(G) char N/\Phi(G)\trianglelefteq G/\Phi(G)$, which implies $P\Phi(G)/\Phi(G)\trianglelefteq G/\Phi(G)$. Since $P$ is Sylow in $N$ and it is normal in $P\Phi(G)$, $P$ is Sylow in $P\Phi(G)$. Then we can use Frattini argument, with wich $G=N_G(P)P\Phi(G)=N_G(P)\Phi(G)$. 
\end{dem}


\begin{coro}
Let $G$ be a finite group. Then the following are equivalent:
\begin{enumerate}
\item $G$ is nilpotent.
\item $G/\Phi(G)$ is abelian.
\item $G'\leq \Phi(G)$.
\end{enumerate}
\end{coro}
We are particularily interested in the equivalence $1\Leftrightarrow 3$. 
\begin{dem}
$2\Leftrightarrow 3$ is obvious. $2\Rightarrow 1$: $G/\Phi(G)$ abelian implies $G/\Phi(G)$ is nilpotent, so $G$ is nilpotent.

$1\Rightarrow 3$. Let $M_1,\dots, M_k$ be all the max subgroups of $G$. Then $\Phi(G)=M_1\cap\cdots\cap M_k$ and $G/M_i$ is of prime order for all $i$, so $G'\leq M_i$ and therefore $G'\leq M_1\cap\cdots\cap M_k=\Phi(G)$. 
\end{dem}

What can we say about finitely generated nilpotent groups? For finitely generated abelian groups we have the structure theorem. Another property of finitely generated abelian groups if that if $H\leq G\Rightarrow d(H)\leq d(G)\Rightarrow H$ is finitely generated. For finitely generated nilpotent groups only it is only true that every subgroup is finitely generated.


\begin{defi}
A \emph{polycyclic series} of a group $G$ is a series $1=G_0\leq G_1\leq\cdots\leq G_r=G$ with $G_{i-1}\trianglelefteq G_i$ for $i=1,\dots, r$ (a subnormal series) and $G_i/G_{i-1}$ cyclic for every $i$. A group with a polycyclic series is called polycyclic.  
\end{defi}

\begin{ej}
A finitely generated abelian group is polycyclic.
\end{ej}

Note that a polycyclic group is necessarily finitely generated, because it is generated by the representatives of the generators of each quotient. 

\begin{teorema}
Let $G$ be a finitely generated nilpotent group. Then $G$ is polycyclic, more precisely, $G$ has a polycyclic and central series. 
\end{teorema}
\begin{dem}
Consider the LCS of $G$: $1=\gamma_{c+1}(G)\leq \gamma_c(G)\leq\cdots\leq\gamma_1(G)=G$. A refinement of a central series is central, so we are going to find the polycyclic series as a refinement of the LCS. Concentrate of step $\gamma_{i+1}(G)\leq\gamma_i(G)$. If $G$ is finitely generated, then all $\gamma_i(G)/\gamma_{i+1}(G)$ are finitely generated and in this case they are abelian. This implies that there is a polycyclic series for $\gamma_i(G)/\gamma_{i+1}(G)$ whose groups take the form of a quotient by $\gamma_{i+1}(G)$. Using the correspondence theorem gives us the piece of the polycyclic series between $\gamma_{i+1}(G)\leq\gamma_i(G)$. 
\end{dem}

As with noetherian rings, one can prove the following:
\begin{teorema}
Let $G$ be a group. Then the following conditions are equivalent. 
\begin{enumerate}
\item Every subgroup of $G$ is finitely generated. 
\item All ascending series of subgroups of $G$ stabilize.
\item Any non-empty set of subgroups of $G$ contains a subgroup that is maximal in the set. 
\end{enumerate}
\end{teorema}

If a group satisgies these conditions, we say that it \emph{satisfies max} (max=the maximal condition on subgroups). Our goal is to prove that finitely generated nilpotent groups satisfy max. 

\begin{lemma}
Max is preserved under extensions: if $N\trianglelefteq G$, and if $N$ and $G/N$ satisfies max, then $G$ sarisfies max.  
\end{lemma}
\begin{proof}
Consider $H\leq G$. Then $H\cap N$ and $HN/N$ are finitely generated. But $H/(H\cap N)\cong HN/N$. Then $H$ is finitely generated ($1\leq H\cap N\leq H$).
\end{proof}

\begin{teorema}
Polycyclic groups satisfy max, and so in particular finitely generated nilpotent groups satisfy max.
\end{teorema}
\begin{dem}
Consider a polycyclic series of $G$: $1=G_0\leq G_i\leq\cdots\leq G_r=G$. Every cquotient is cyclic so trivially satisfies max. Then every group of the series satisfies max, in particular $G_r=G$. 
\end{dem}

Since finitely generated nilpotent groups are polycyclic, by taking generators $u_1,\dots, u_n$ in every cyclic quotient:
\[
G=G_1\overset{u_1}{\geq} G_2\geq\cdots\geq G_n\overset{u_n}{\geq} G_{n+1}=1
\]
every element can be uniquely written as $g=u_1^{\alpha_1}\cdots u_n^{\alpha_n}$ with $\alpha_i\in \Z$ if $o(u_i)=\infty$ and $0\leq\alpha_i\leq o(u_i)$ otherwise. How do they multiply together? We are going to do it when all $u_i$ are of infinite order. 

\begin{defi}
Let $G$ be a group.
\begin{enumerate}
\item An element $g\in G$ is a \emph{torsion (periodic) element} if $o(g)<\infty$. 
\item $G$ is \emph{torsion-free} if the only torsion element is the identity.
\end{enumerate}

For example, a finitely generated abelian group $G$ is torsion-free iff $G\cong C_\infty\times\cdots\times C_\infty$. 
\end{defi}

\begin{lemma}
Being torsion-free is preserved by extensions: if $N\trianglelefteq G$ and if $N$ and $G/N$ are torsion free, then $G$ is torsion free.
\end{lemma}
\begin{proof}
We have to see that if $g\in G$ is of finite order, then $g=1$. Let $g$ of order $n<\infty$, then $g^n=1$ and so $\overline{g}^n=\overline{1}\in G/N$, so $\overline{g}=\overline{1}$, hence $g$ is in $N$ and of finite order, therefore $g=1$. 
\end{proof}

How can we know if a nilpotent group is torsion-free.

\begin{teorema}
If $G$ is nilpotent then: $G$ torsion-free iff $Z(G)$ is torsion-free.
\end{teorema}
We need
\begin{lemma}Let $G$ be a group. Then: $Z(G)$ is torsion free iff $Z_2(G)/Z(G)$ is torsion-free. 
\end{lemma}
\begin{proof}
Take $\overline{g}\in Z_2(G)/Z(G)$ of order $n<\infty$. $\overline{g}^n=\overline{1}\Rightarrow g^n\in Z(G)\Rightarrow [g^n,x]=1\forall x\in G$. Imagine that we can take $n$ out of the commutator. Since $[g,x]\in Z(G)$ (because the UCS is central) and $[g,x]^n=1$, $[g,x]$ is of finite order in $Z(G)$, $[g,x]=1$ for all $x\in G$ and therefore $g\in Z(G)$, which means $\overline{g}=\overline{1}$. The sufficient condicion to do this is that $g$ commutes with the $[g,x]$, but this is true because $[g,x]\in Z(G)$.
\end{proof}

\begin{dem}[of the theorem]
The implication to the right is trivial. By using the last lemma and induction, every quotient in the UCS is torsion-free for $i=1,\dots, c$, where $c$ is the nilpotency class ($Z_c(G)=G$) (torsion-free is preserved by extension). By using the first lemma $G$ is torsion free.
\end{dem}

\begin{teorema}
Let $G$ be a finitely generated nilpotent group and assume that $G$ is torsion-free. Then consider a polycyclic and central series which refines the UCS
$$G=G_1\overset{u_1}{\geq} G_2\geq\cdots\geq G_n\overset{u_n}{\geq} G_{n+1}=1$$
Then:
\begin{enumerate}
\item All factors $G_i/G_{i+1}$ are inifinite cyclic, so if we choose a generator $u_i$ for every of these quotients, then every $g\in G$ can be uniquely written as $g=u_1^{\alpha_1}\cdots u_n^{\alpha_n}$ with $\alpha_i\in\Z$.
\item Two elements $g=u_1^{\alpha_1}\cdots u_n^{\alpha_n}$ and $h=u_1^{\beta_1}\cdots u_n^{\beta_n}$ are multiplied as follows:
$$gh=u_1^{\gamma_1}\cdots u_n^{\gamma_n}$$
where $\gamma_i=f_i(\alpha_1,\dots, \alpha_n,\beta_1,\dots,\beta_n)$ with $f_i\in\Q[x_1,\dots, x_n,y_1,\dots, y_n]$.
\end{enumerate}
\end{teorema}

\begin{nota}
The polynomials $f_i$ have rational coefficients byt always take integer values over the intengers because $\gamma_i$ has to be in $\Z$. This does not mean that the coefficients of $f_i$ are integers. For example, the polynomial in one variable $f(x)=\frac{1}{2}x-\frac{1}{2}x^2$. If $n\in\Z$ then $f(n)=\frac{n(n-1)}{2}=\binom{n}{2}\in\Z$. The same for $f(x)=\frac{x(x-1)\cdots (x-k+1)}{k!}$. It can be proved that $f_i$ is a polynomial is a polynomial with in integers coefficients in the binomial coefficientes $\binom{x_i}{k_i}, \binom{y_i}{k_i}$ for $i=1,\dots, n$ and $k_i\in\N$. 
\end{nota}


ADD TO BIBLIOGRAPHY
Basic references: D. Robinson, \emph{A course in the theory of groups}. M. Suzuk, \emph{Group Theory} I and II. A.E. Clement, S. Majewicz, M. Zyman, \emph{The theory of nilpotent groups} (proof of the multiplication rule of the las theorem on page 118 -a result that was proved by P. Hall).

\section{$p$-Nilpotency}

\begin{defi}
Let $\Pi$ be a set of primes. We say that a finite group $G$ is an $\Pi$-group if for all prime $p$ divisor of $|G|$, $p\in \Pi$. 
\end{defi}

\begin{defi}
$\Pi'=\{p$ prime $\mid p\notin \Pi\}$, $p'=\{q$ prime $\mid q\neq p\}$. 
\end{defi}

\begin{defi}
$Hall_H(G)=\{H\leq G\mid H$ is a $\Pi$-group and $|G:H|$ is an $\Pi'$-number$\}$.
\end{defi}

\begin{defi}
Let $G$ a finite group and $\Pi$ a set of primes. A $\Pi$-subroups is a Hall $\Pi$-subgroup if $p||G:H|$ implies $p\notin \Pi$. 
\end{defi}

\begin{prop}
Let $G$ a finite group and $H$ a Hall $\Pi$-subgroup. Then:

 If $N\trianglelefteq G$ then $HN/N$ is a Hall $\Pi$-subgroup of $G/N$ and $N\cap H$ is of $N$.

\end{prop}

\begin{defi}
Let $G$ be a (finite) group and $H\leq G$. We say that $K\leq G$ is a \emph{complement} of $H$ in $G$ if $G=HK$ and $K\cap H=1$. 
\end{defi}


\begin{defi}
Let $G$ be a finite group. A (normal) $p$-\emph{complement} is a (normal) complement of a Sylow $p$-subgroup.
\end{defi}

\begin{defi}
A finite group $G$ is $p$-nilpotent if it has a normal complement. 
\end{defi}

If $S\in Syl_p(G)$ and $N$ is a normal $p$-complement, $G=S\ltimes N$:

\begin{ej}\
\begin{enumerate}
\item $p\not| |G|$, $N=G$. If $G$ is a $p$-group, $N=1$.
\item If $G$ is abelian, then $G$ is $p$-nilpotent for all primes. 
\item If $G$ is nilpotent, $G\cong S_{p_1}\times\cdots\times S_{p_k}$ and $N_1=1\times S_{p_2}\times\cdots\times S_{p_k}=1$. Then $G$ is $p_1$-nilpotent.
\item If $G$ is non-abelian a simple group and $p||G|$, then $G$ is not $p$-nilpotent. 
\item $D_{2p}$ where $p$ is an odd prime. Then $D_{2p}$ is 2-nilpotent and not $p$-nilpotent.  
\end{enumerate}
\end{ej}


\begin{prop}\

The class of $p$-nilpotent group is closed under direct products, subgroups and quotient. It is not closed under extensions.



\end{prop}
\begin{dem}
Closed under products is trivial.

Closed under subgroups: let $H\leq G$ and $N$ a normal $p$-complement. Set $K=H\cap N$, which is normal in $H$. It is clear that $p$ does not divide $|K|$ since it doesn't divide $|N|$. Recall that $H/K=H/H\cap N\cong HN/N\leq G/N\cong S$, where $S$ is a Sylow $p$-subgroup of $G$.

Closed under quotients: Let $H\trianglelefteq G$ and $N$ a normal $p$-complement. Set $K=NH/H\trianglelefteq G/H$. We want to show that the index of this subgroup is a power of $p$ and its order is coprime with $p$. We have by the third isomorphism theorem $(G/H)/(NH/N)\cong G/NH$, which is a quotient of $G/N$. Now, using $NH/H\cong N/H\cap N$ we are done.  

Not closed under extensions: $1\to C_p\to D_{2p}\to C_2\to 1$.
\end{dem}

\begin{teorema}
Let $G$ be a finite group. Then the following conditions are equivalent:
\begin{enumerate}
\item $G$ is nilpotent.
\item $G\cong S_{p_1}\times\cdots\times S_{p_k}$, where $S_{p_1}$ are the Sylow $p$-subgroups of $G$.
\item For every prime $p$, the Sylow $p$-subgroups are normal. 
\item $G$ is $p$-nilpotent for every prime $p$.
\end{enumerate}
\end{teorema}
\begin{dem}
$2\Leftrightarrow 3$ is trivial (recall that the intersection of the Sylow $p$-groups is trivial, so that they form an internal direct product). $1\Leftrightarrow 2$ was proved earlier in the course.

Let us prove $2\Rightarrow 4$. Let $p$ be a prime and define $N_p=\prod_{q\neq p} S_q$ where $S_q$ is a Sylow $q$-subgroup. 

Finaly we prove $4\Rightarrow 3$. Let $p$ be a prime and define $K=\bigcap_{q\neq p} N_q$. Then $K$ is a normal subgroup and it is a $p$-group, because if one takes any prime different from $p$, it does not divide the order of some $N_q$, so it doesn't divide $|K|$. Since $N_q$ is normal and contains a Sylow subgroup, it contains every Sylow subgroup, and we are done.
\end{dem}

\begin{prop}
Let $G$ be a finite group. Then the following are equivalent.
\begin{enumerate}
\item $G$ is $p$-nilpotent.
\item $K=\{g\in G\mid \gcd(p,o(g))=1\}=\{g\in G\mid g$ is a $p'$-element$\}$ is a subgroup.
\end{enumerate}
\end{prop}
\begin{dem}
$1\Rightarrow 2$. Let $N$ be a normal $p$-complement. We have $N\subseteq K$ by definition. We prove the other inclusion by contradiction, assume $x\in K\N$. Take $H=\gene{x,N}$. Then $H/N=\gene{\overline{n}}\leq G/N$ is a $p$-group, contradiction.

$2\Rightarrow 1$. Let $K\leq G$ and assume $K$ is a $p'$-group and it is normal (in fact it is characteristic). Now observe that $G/K$ is a $p$-group by Cauchy theorem.
\end{dem}

\section{Transfer}
\begin{defi}
$G^{ab}=G/[G,G]$ is the largest abelian quotient of $G$.
\end{defi}
If $H\leq G$, the natural inclusion $i:H\hookrightarrow G$ induces an homomorphism $res: H^{ab}\to G^{ab}$, $res(h[H,H])=h[G,G]$. This map is the composition of the maps $H/[H,H]\to H/[G,G]\cap H\to H[G,G]/[G,G]$ where the first map is an epimorphism. Is there an homomorphism $\varphi:G^{ab}\to H^{ab}$?


Let $T=\{t_1,\dots, t_n\}$ be a left-transversal of $H$ in $G$ and assume $|G:H|=n$. Write $G=\coprod_{i=1}^nH$. We have that $\{t_iH\}_{i=1}^n$ is a partition of $G$. Now $xT$ is a transversal, so $\{xt_iH\}=\{t_iH\}$. Then, there exists a permutation $\sigma\in S_n$ such that for all $i$, $xt_iH=t_{\sigma(i)}H$. Hence, $xt_i=t_{\sigma(i)}h_{\sigma(i)}$ for some $h_{\sigma(i)}\in H$. More precisely, $h_{\sigma(i)}=t_{\sigma(i)}^{-1}xt_i$. 
\begin{defi}
We define the \emph{transfer} of $H$ in $G$ as $V:G\to H^{ab}$ by $V(x)=\prod_{i=1}^nh_{\sigma(i)}[H,H]$. 
\end{defi}

\begin{teorema}
Let $H\leq G$, $|G:H|<\infty$ and $V:G\to H^{ab}$ the transfer. Then
\begin{enumerate}
\item $V$ does not depend on the transversal. 
\item $V$ is a homomorphism of group. 
\end{enumerate}
\end{teorema}

\begin{coro}
$[G,]\leq \ker V$. 
\end{coro}

Take $x\in G$. Then $\gene{x}$ acts by left multiplication on $G/H=\{t_iH\}_{i^1}^n$. We have 
\[
t_1H,xt_1H, x^2t_1H,\dots, x^{n_1}t_1H=t_1H
\]
Not all cosets are necessarily reached with $t_1$, but we can do this until we found $t_s$ such that $x^{n_s}t_sH=t_sH$ with which $\{t_1,\dots, t_s\}$ is enough to generate all cosets  by the action of $x$. We then choose this set as a transversal. Now we have to compute the value of $h$. For $t_1H$, multiplication by $x$ yields $xt_1H$, so $h=1$. The same if we multiply again by $x$,, until $x^{n_1-1}t_1H$ is transformed to $t_1H$, where $h=t_1^{-1}x^{n_1}t_1=(x^{n_1})^{t_1}$. Hence we have the theorem
\begin{teorema}
$V(x)=(x^{n_1})^{t_1}\cdots (x^{n_s})^{t_s}$ where $n_1+\cdots+n_s=|G:H|$.
\end{teorema}

\begin{teorema}[Atiyah]
Let $G$ be a finite group and $S$ a Sylow $p$-subgroup. Then the following are equivalent:
\begin{enumerate}
\item $G$ is $p$-nilpotent.
\item For all $i$ big enough, $H^i(G,\F_p)\cong H^i(S,\F_p)$. 
\end{enumerate}
\end{teorema}



\begin{teorema}
Let $G$ be a group and $N\trianglelefteq G$ of finite index such that $N\leq Z(G)$. Then $V(x)=x^n$ with $n=|G:N|$.
 \end{teorema}
 \begin{dem}
 $(x^{n_1})^{t_1}\in N\leq Z(G)$. Then $(x^{n_1})^{t_1}=))x^{n_1})^{t_1})^{t_1^{-1}}=x^{n_1}$. $V(x)=x^{n_1}\cdots x^{n_s}=x^{n_1+\cdots+n_s}=x^n.$
 \end{dem}
 
 \begin{teorema}
Let $G$ be a group and suppose that $|G;Z(G)|=n<\infty$. Then $G'$ is finitely generated and $\exp(G')|n$. In particular, $G'$ is finite. 
\end{teorema}
\begin{dem}
$G'=\gene{[g,h]\mid g,h\in G}$. $G=\bigcup t_iZ(G)$. Take $[t_iz_i, t_jz_j]=[t_1,t_2]$ with $z_i,z_j\in Z(G)$. Then $G'=\gene{[t_i,t_j]\mid t_i,t_j\in T}$, so it is finitely generated. 

Now apply the previous theorem to $V:G\to Z(G)$, which sends $g\mapsto g^n$. We know that $G'\leq \ker V$. This means that for all $g\in G'$, $g^n=1$, which implies the conclusion of the theorem. 
\end{dem}

\begin{teorema}
Let $G$ be a finite group and $S$ an abelian Sylow $p$-sugbgroup. Then the following are equivalent.
\begin{enumerate}
\item $G$ is $p$-nilpotent.
\item If $x,y\in S$ such that $x\sim^G y$ ($x$ is conjugated to $y$ in $G$), then $x=y$. 
\item $V:G\to S$ satisfies that for all $s\in S$, $V(s)=s^n$ where $n=|G:S|$. 
\end{enumerate}
\end{teorema}

\begin{dem}
$1\Rightarrow 2$. Then $G$ as a normal $p$-complement $N$. If $x\sim^G y$, then $\exists g\in G$ such that $y=x^g$. $G=SN$, so $g=sn$ with $s\in S$, $n\in N$. Hence $y=x^{sn}=(x^s)^n=x^n$ since the Sylow is abelian. Multiply by $x^{-1}$ on both sides to get $S\ni x^{-1}y=x^{-1}x^n=[x,n]\in[S,N]\leq N$, so it is in $N\cap S=1$ and folloes that $x=y$. 

$2\Rightarrow 2$. If $x\in G$ and $V:G\to S$, we know that $V(x)=(x^{n_1})^{t_1}\cdots(x^{n_s})^{t_s}$, where every $x^{n_i}\in S$. If $x\in S$, then $V(X)$ by 2 we conclude that $(x^{n_1})^{t_i}=x^{n_i}$, and therefore $V(x)=x^{n_1+\cdots +n_s}=x^n$. 

$3\Rightarrow 1$. If we prove that $V$ is surjective, we will get $S=G/\ker V$ and if the we prove that $\ker V\cap S=1$, then $\ker V$ is a complement. Let us focus on $V|_{S}:S\to S$, $s\to s^n$. Since $S$ is a $p$-group and $\gcd(p,n)=1$, $\ker V|_{S}=1$. Hence $\ker V\cap S=1$. Since the kernel is trivial and both domain and codomain have the same order, $V$ is an isomorphism when restricted to $S$. Then $\ker V$ is a normal $p$-complement of $S$ in $G$. 
\end{dem}

\begin{teorema}[Burnside]
Let $G$ be a finite group and $S$ an abelian Sylow $p$-subgroup. If $N_G(S)=C_G(S)$ (this actually implies that $S$ is abelian because $S\leq N_G(S)$) then $G$ is $p$-nilpotent. 
\end{teorema}
\begin{dem}
Let $x,y\in S$ with $x\sim^G y$. We prove that $x=y$. There exists $g\in G$ such that $y=x^g$. This means that $y\in S$ and $y\in S^g$. Both are contained in $C_G(y)$. Since $S,S^g\in Syl_p(C_G(y))$, they are conjugated, i.e. for some $z\in C_G(y)$, $S=S^{gz}$. Then $gz\in N_G(S)=C_G(S)$. So $y=x^g\Rightarrow y=y^z=x^{gz}=x$. By the previous theorem, $G$ is nilpotent.
\end{dem}


\begin{coro}
Let $G$ be a finite group and $|G|=p_1\cdots p_n$ (free of squares). Then $G$ is solvable. 
\end{coro}
\begin{dem}
Let $S\leq G$ and $\varphi:N_G(S)\to Aut(S)$, with $\ker\varphi=C_G(S)$. We have $N_G(S)/C_G(S)\hookrightarrow Aut(S)$. We can assume $p_1<\cdots<p_n$ and $S_{p_1}=C_{p_1}$. Since $|Aut(C_{p_1})|=p_1-1$, the map $N_G(S)/C_G(S)\to Aut(S)$ is the trivial one, so $N_G(S)=C_G(S)$, and then $G$ is a $p_1$-nilpotent group, implying that $G\cong N_{p_1}\cong C_{p_1}\ltimes N_{p_1}$. By induction we prove it for all primes divisores of $|G|$.

\end{dem}


\begin{teorema}
Let $G$ be a finite group and $S$ a Sylow $p$-subgroup. Then the following conditions are equivalent.
\begin{enumerate}
\item $G$ is $p$-nilpotent.
\item If $x,y\in S$ and $x\sim^G y$, then $x\sim^S y$.
\end{enumerate}
\end{teorema}

\begin{dem}
$1\Rightarrow 2$. If $x\sim^G y$, there exists $g\in G$ such that $y=x^g$. Let $N$ be a normal $p$-complement. Then $G=SN$. We have $y=x^{sn}=(x^s)^n$ with $s\in S$ and $n\in N$. Then $S\ni x^{-s}y=x^{-s}(x^s)^n=[x^s,n]\in N$. Since $S\cap N=1$, $x^s=y$. 

$2\Rightarrow 1$. We apply the proposition next to this theorem. We are going to construct a sequence of nromal subgroups $N_i$ such that $G=SN_i$ and $S\cap N_i=\gamma_i(S)$. In particular, if $c-1$ is the nilpotency class of $S$, we have $G=SN_c$ and $S\cap N_c=1$. In particular, $G$ is $p$-nilpotent and $N_i$ is the normal $p$-complement. We argue by induction: $\gamma_1(S)=2$, $G=N_1$, $G=SG$, $S=S\cap G$. Supoose tthe claim is true for $i$ and check it for $i+1$. We want to apply the proposition to $G=N_i$ and $J=\gamma_{i+1}(S)$. We have $\gamma_i(S)\in Syl_p(N_i)$. Let $x,y\in \gamma_i(S)=S\cap N_i$ with $x\sim^{N_i} y$. Then $x\sim^G y$, and by hypothesis $x\sim^S y$. This means that there exists $s\in S$ such that $y=x^s$, but $x\gamma_{i+1}(S)=y\gamma_{i+1}(S)$. By the proposition there exists $K\trianglelefteq N_i$ such that $N_i=\gamma_i(S)K$ and $\gamma_i(S)\cap K=\gamma_{i+1}(S)$. We have $KS=K\gamma_i(S)S=N_iS=G$ and $K\cap S=K\cap N_i\cap S=K\cap\gamma_i(S)=\gamma_{i+1}(S)$. The conditions are fulfilled with $N_{i+1}=K$, except that we need $K$ to be normal in $G$. Take a conjugate $K^g$. We have $K^g\trianglelefteq N_i^g=N_i$, so $KK^g\leq N_i$. Note that $p'=|K:J|$ is coprime with $p$ and so is $|N_i: S\cap N_i|$. We prove that $|K^g : K\cap K^g|$ is a power of $p'$ and coprime with $p$. $|K^gK:J|=\frac{|K|}{|J|}\frac{|K^g|}{|K\cap K^g|}$. Then $K^g=K^s$ and $|K^s:J^s|=|K:J|$

DIBUJO DE DIAGRAMA EN ALGUNA PARTE PARA HACER LOS DE LOS ÍNDICES CON $N_i$ ON TOP, DEBAJO $S\cap N_i$ Y $K$ (EL ÚLTIMO TIENE EN MEDIO $KK^g$ Y ABAJO DEL ROMBO ESTÁ $J$, QUE YA VA A 1. 
\end{dem}
For $p$ odd it is enough to consider elements of order $p$. 
\begin{prop}
Let $G$ be a finite group, $S$ a Sylow $p$-subgroup of $G$ and $J\trianglelefteq S$ such that $S/J$ is abelian. Then the following are equivalent.
\begin{enumerate}
\item There exists $N\trianglelefteq G$ such that $G=SN$ and $S\cap N=J$. 
\item $x,y\in S$, $x\simeq^G y\Rightarrow xJ=yJ$. 
\end{enumerate}
\end{prop}
\begin{dem}
$1\Rightarrow 2$. Let $x\simeq^G y$ with $x,y\in S$. Then $\exists g\in G$ such that $y=x^g$ with $g=sn$ with $s\in S$ and $n\in N$. Then $y=x^g=x^{sn}$, $x^{-s}y=[x^s,n]\in J$, then $yJ=x^sJ$ (this implies that $G/J$ is abelian since conjugation is trivial).

$2\Rightarrow 1$. Define the transfer $V:G\to S^{ab}=S/S'\twoheadrightarrow S/J$ since $J\subseteq S'$. We have the formula $V(x)=(x^{n_1})^{t_1}\cdots(x^{n_s})^{t_s}J$. Then $S\ni x^{n_i}\sim^G  (x^{n_i})^{t_i}\in S$, meaning that $x^{n_1}J=(x^{n_1})^{t_i}J$. This means that $V(x)=x^{n_1}\cdots x^{n_s}J$. Let $N=\ker V$, then we have $S\cap N=J$ and $SN=G$. 
\end{dem}

\begin{teorema}
Let $G$ be a group. Then the following are equivalent:
\begin{enumerate}
\item $G$ is $p$-nilpotent.
\item $\forall P\leq G$ $p$-subgroup, $N_G(P)/C_G(P)$ is a $p$-group.
\item $\forall P\leq G$ $p$-subgroup, $N_G(P)$ is $p$-nilpotent.
\end{enumerate}
\end{teorema}

\begin{teorema}[Thompson]
Let $G$ be a finite group and $\alpha_a$ a regular automorphism (the only fixed element is the identity) of prime order. Then $G$ is $p$-nilpotent.
\end{teorema}

\begin{teorema}[Tate]
Let $G$ be a finite group a $S$ a Sylow $p$-subgroup. Then the following are equivalent.
\begin{enumerate}
\item $G$ is $p$-nilpotent.
\item $G/([G,G]G^p)\cong S/([S,S]S^p)$.
\end{enumerate}
\end{teorema}

\section{Solvable groups}

\begin{defi}
Let $G$ be a group. Then we define $G^{(1)}=G'$ and $G^{(n)}=(G^{(n-1)})'$. The series formed by thse groups is called the \emph{derived series}.
\end{defi}

Note that $G^{(2)}=[[G,G],[G,G]]=[\gamma_2(G),\gamma_2(G)]\leq \gamma_4(G)$. Similarily, $G^{(3)}\leq \gamma_8(G)$. In general

\begin{lemma}
For all $n\geq 1$, $G^{(n)}\leq \gamma_{2^n}(G)$.
\end{lemma}

\begin{defi}
Let $G$ be a group. We say that $G$ is \emph{solvable} if there exists $n\geq 1$ such that $G^{(n)}=1$.
\end{defi}

\begin{prop}\
\begin{enumerate}
\item Solvability is closed under subgroups.
\item It is closed under quotients. 
\end{enumerate}
\end{prop}


\begin{lemma}
Let $1\to H\to G\to \overline{G}\to 1$ an extension of groups. Then $G$ is solvable iff $N$ and $\overline{G}$ are solvable.
\end{lemma}
\begin{proof}
We only need to show the implication to the left. We know that there exist $n,m\geq 1$ such that $\overline{G}^{(n)}=\overline{1}$ and $N^{(m)}=1$. Then $G^{(n)}\leq N$. Then $(G^{(n)})^{(m)}\leq N^{(m)}=1$.
\end{proof}

\begin{prop}
Let $G$ finite be a group. Then the following are equivalent.
\begin{enumerate}
\item $G$ is solvable.
\item There exists a series of subgroups $N_0=1\leq N_1\leq N_2\leq\cdots\leq N_s=G$ such that $N_i\trianglelefteq N_{i+1}$ and $N_{i+1}/N_i$ is abelian.
\item There exists a series of subgroups $N_0=1\leq N_1\leq N_2\leq\cdots\leq N_r=G$ such that $N_i\trianglelefteq N_{i+1}$ and $N_{i+1}/N_i$ is cyclic. 
\end{enumerate}
\end{prop}

\begin{dem}
$1\Rightarrow 2$ is trivial. 

$2\Rightarrow 3$. We can obtain a cyclic series from a series of finite abelian groups doing quotients.

$3\Rightarrow 1$ is trivial. 
\end{dem}

\begin{ej}\
\begin{enumerate}
\item Abelian groups and nilpotent groups are solvable.
\item $D_{2n}$ is solvable for all $n$. Write $D_{2n}=\gene{a,b\mid a^n=b^2=1, [a,b]=a^{-2}}$. Then $D_{2n}'=\gene{a^2}$ and $D_{2n}^{(2)}=1$.
\item What happens with $S_n$? $S_2=C_2$ so it is solvable. $S_3=D_6$ which is solvable. We have computed that $(S_5)'=A_5$ and we knoe that $S_5$ is a non-abelian simple group, so $A_5'=A_5$ and it is not solvable (the same works for $S_n$ with $n>5$). 
It is also true that $S_4'=A_4=\{(abc),(ab)(cd)\mid a,b,c,d\in\{1,\dots, n\}\}\cong C_3\ltimes (C_2\times C_2)$. But in this case $A_4'=V\cong C_2\times C_2$ and so it is solvable. 
\end{enumerate}
\end{ej}

Let $f\in\Q[x]$ of degree $n$. Denote $L=\Q(f)=\Q[\alpha\mid f(\alpha)=0]$. Now $Gal(L:\Q)\hookrightarrow S_n$. 

\begin{defi}
$L:\Q$ is a \emph{radical extension} if there exist $\Q=K_1\leq K_2\leq\cdots\leq K_n=L$ such that $K_i=K_{i-1}[\sqrt[n_i]{a}]$. 
\end{defi}

\begin{teorema}
$\Q(f):\Q$ is a radical extension iff $Gal(f)=Gal(\Q(f):\Q)$ is solvable. 
\end{teorema}


\begin{lemma}
Let $G$ be an irreducible polynomial of degree 5 with exactly three real roots. Then $Gal(f)=Gal(\Q(f):\Q)\cong S_5$.
\end{lemma}
\begin{proof}
 Since $G$ is irreducible, $(12345)\in Gal(f)$. Since $f$ has 3 real roots, the complex conjugation is the cycle $(ij)$. Now $\gene{(12345)(ij)}=S_5$. 
\end{proof}

\begin{ej}
$f(x)=x^5-4x+2$. This polynomial has exactly 3 real roots. 
\end{ej}

\begin{teorema}[Zassenhaus, Schur]
Let $G$ be a finite group and $N\trianglelefteq G$ such that $\gcd(|N|,|G/N|)=1$. Then $N$ has a complement in $G$. Furtheremore, assume that $N$ or $G/N$ is solvable. Then any two complements of $N$ in $G$ are conjugated in $G$.
\end{teorema}

Note that if 2 divides $|N|$, then $|G/N|$ is odd and viceversa, so in this case, one of them is always solvable because groups of odd order are solvable. 

\begin{dem}
By induction on the order of $G$. Easy one, $N$ is solvable. $N'<N$ ($N'N^p<N$). If $N'=1$ apply the next theorem by Zassenhaus. If $N'\neq 1$ we can apply induction, so if $H_1$ and $H_2$ are two complements of $N$ in $G$, then $H_1N'/N'$ and $H_2N'/N'$ are two complements of $N/N'$ in $G/N'$. Now $H_1N'/N'\cong H_1/(N'\cap H_1)\cong H_1$. Since any two complements are conjugate, there exists $g\in G$ such that $H_1N'/N'=(H_2N'/N')^g=1$, so $H_1N'=(H_2N')^g=H_2^gN'$. Both $H_1$ are $H_2^g$ are complements of $N'$ in $H_1N'$. Recall that $|H_1N'|=|H_1||N'|<|H_1||N|=|G|$, so we can apply the induction hypothesis, by which there exists $x\in H_1N'$ such that $H_1=((H_2)^g)^x=H_2^{gx}$. 

Now assume that $G/N$ is solvable. Take $M/N$ a minimal normal subgroup of $G/N$. If $M/N=G/N$, then $G/N$ is elementary abelian by the following lemma.

\begin{lemma}
Let $G$ be a solvable group and $N$ a minimal normal subgroup. Then $N$ is elementary abelian for some prime $p$. 
\end{lemma}
\begin{proof}
Take $p||N^{ab}$, then $N'N^p$ is a normal subgroup of $G$. By minimality $N'N^p=1$. 
\end{proof}

It follows that the complement is a Sylow $p$-subgroup and this case follows from the Sylow theory. 

If $M/N<G/N$. Then consider $B_1$ and $B_2$ two complements of $N$ in $G$. %Then $M=M\cap G=M\cap B_iN=(MN\cap B_i)N=(M\cap B_i)N$ (using that $N\leq M$ and Dedekind law)
Then $M=M\cap G=M\cap B_iN=(M\cap B_i)N$ (using that $N\leq M$ and Dedekind law). So $M\cap B_i$ is a complement of $N$ in $M$. Therefore $M\cap B_1$ and $M\cap B_2$ are conjugate. Since $M$ is normal this means that $M\cap B_1=M\cap B_2^g$. Take $T=M\cap B_1$. $T$ is no the trivial group, since $T=1$ implies that $|MB_1|=|M||B_1|>|N||B_1|=|G|$. We clearly have that $T\trianglelefteq B_1,B_2$. $N_G(T)=(N_G(T)\cap G)=(N_G(T)\cap NB_1)=(N_G(T)\cap N)B_1$. Also $N_G(T)=(N_G(T)\cap N)B_2^g$. Hence $(N_G(T)\cap N)T\cap B_1=N_G(T)\cap NT\cap B_1$ and we apply Dedekind law to get $(N_G(T)T\cap NT\cap B_1T)=(N_G(T)\cap N\cap B_1)T=T$. This means that $B_1/T$ is a complement of $(N_G(T)\cap N)T/T$ in $N_G(T)/T$. Since $T$ is not trivial we can apply induction. Similarily with $B_2^G/T$. Then there exists $z\in N_G(T)$ such that $B_1=(B_2^g)^z=B_2^{gz}$. 
\end{dem}

\begin{defi}
$H\leq G$ is a $\Pi$-Hall subgroup if $p||H|$ implies $p\in \Pi$ and $p||G:H|$ implies $p\notin p\notin \Pi$.
\end{defi}

\begin{teorema}
Let $G$ be a solvable grup and $\Pi$ a set of primes. Then $G$ contains $\Pi$-Hall subgroups and any of two of them are conjugated in $G$. 
\end{teorema}

Consider $1\to N\to G\to G/N\to 1$  with $N$ \emph{small} ($N$ abelian or elementary abelian for a prime $p$). $G$ acts on $N$ by conjugation. $N$ acts trivially on $N$ by conjugation. We have an action by conjugation of $G/N$ in $N$. Call $\varphi:G\to Aut(N)$ the action given by $g\mapsto G_g:N\to N$ such that $C_g(n)=g^{-1}ng$. It is clear that $\ker\varphi=C_G(N)\geq N$ since $N$ is abelian. 

Now suppose $N$ has a complement $H$ in $G$. Then $H=H/(H\cap N)\cong HN/N=G/N$. Hence the extension is split, because we have an homomorphism $\phi:G/N\to G$ such that $\pi\circ\phi=Id$ where $\pi$ is the quotient map (eventually this is an iff). We have a section at the level of sets $s:G/N\to G$ s.t. $\{s(\overline{g})\mid \overline{g}\in G/N\}$ is a transversal and $\overline{g}=s(\overline{g})N$. When is $s$ a group homomorphism? 

An example when this ispossible is $1\to C_p \to C_p\times C_p\to C_p\to 1$ and one where it is not is $1\to C_p\to C_{p^2}\to C_p\to 1$. In the first case we can always form transversal that is not a subgroup: if $T=1\times C_p$ we can define $\widetilde{T}=\{(1,x)\mid x\neq 1\}\cup \{x,1\}$. In general we only need to substitute the identity element by a different one.

For $N\trianglelefteq G$, take a transversal $T\subseteq G$. We have $TN/N=G/N$, so we define $s:\overline{G}G/N\to G$ by $\overline{g}\mapsto s(\overline{s})=t$, where $t$ is the only $t\in T$ such that $tN=\overline{g}$.

In general $s$ is not a homomorphism. When is it? How about when $T$ is a subgroup? What we have in general is that $s(\overline{g}_1)s(\overline{g}_2)s(\overline{g_1g_2})=\alpha(\overline{g}_1,\overline{g}_2)\in N$, where $\alpha:\overline{G}\times\overline{G}\to N$. This $\alpha$ satisfies for all $\overline{g}_1,\overline{g}_2,\overline{g}_3\in G$ 
\begin{equation}\label{identity}
\alpha(\overline{g}_1,\overline{g}_2\overline{g}_3)\alpha(\overline{g}_2,\overline{g}_3)=\alpha(\overline{g}_1\overline{g}_2,\overline{g}_3)\alpha(\overline{g}_1,\overline{g}_2)^{\overline{g}_3}
\end{equation}
In addition, given such $\alpha$ we can recover the transversal. 

Given a group $G$, $N$ an abelian group and an action of $\overline{G}$ in $N$, we define the \emph{cocycles} $Z^2(G,N)=\{\alpha:\overline{G}\times\overline{G}\to N\mid \alpha$ satisfies \ref{identity}$\}$ and the \emph{coboundaries} $B^2(G,N)=\{\alpha:overline{G}\times\overline{G}\to N\mid\exists\beta:\overline{G}\to N\ s.t.\ \alpha(\overline{x},\overline{y})=\beta(\overline{y})^{\overline{x}}\beta(\overline{x},\overline{y})\beta(\overline{x})\}$

One can see that the \emph{cohomology group} $H^2(G,N)=Z^2(G,N)/B^2(G,N)$ is well defined. 

Let $N$ be an elementary abelian $p$-group. We know that $N$ is a vector space over $\F_p$. Then $N$ becomes a $\F_p[\overline{G}]$-module. In this situation, using representation theory we get that $\F_p[G]$ is semisimple and $H^2(G,N)=0$. 

We have proved that if $G$ is a finite group and $N\trianglelefteq G$ such that $\gcd(|N|,|G:N|)=1$, then $N$ has a complement $H$ in $G$, which is equivalent to having a transversal of $N$ in $G$ which is a subgroup of $G$. 

\begin{teorema}
Let $G$ be a finite group and $N$ an abelian normal subgroup of $G$ with $\gcd(|N|,|G:N|)=1$. Then:
\begin{enumerate}
\item $N$ has a complement in $G$.
\item Two different complements are conjugated in $G$.
\end{enumerate}
\end{teorema}
\begin{dem}
Take a section $s:\overline{G}\to G$ defined by $s(\overline{g})=t\in T$ with $\overline{g}=tN$. We now define $\beta:\overline{G}\to G$ by $\beta(\overline{g})=\prod_{\overline{x}\in \overline{G}}\alpha(\overline{x},\overline{g})^m$, where $m|G:N|\equiv 1\mod |N|$ (exists because the indices are coprime) and $\alpha$ is the one defined previousle as the error of $s$ being a homomorphism. Define $s':\overline{G}\to G$ by $s'(\overline{g})=s(\overline{g})\beta(\overline{g})$, which is a group homomorphism. The rest is left as an exercie to the reader, who should probably look for this proof in a book.
\end{dem}

\begin{teorema}[Zassenhaus]
Let $G$ be a finite group, $N\trianglelefteq G$ such that $\gcd(|N|,|N:G|)=1$. Then $N$ has a complement in $G$.
\end{teorema}
\begin{dem}
Let $G$ be a minimal counterexample. We claim that $N$ is nilpotent iff for all $p||N|$ and $S\in Syl_p(N)$ $S$ is normal in $N$. This claim follows from this other: for all $p||N|, S\in Syl_p(N9$, $S$ is normal in $G$. Let us prove it. Let $S\in Syl_p(N)=Syl_p(G)$. By the Frattini argument $G=NN_G(S)$. We have $G/N=NN_G(S)/N\cong N_G(S)/(N)\cap N_G(S)$. It follows from equality of orders and the hyptohesis of the theorem that $\gcd(|G:N|,|N\cap N_G(S))=1$, which implies $(|N_G(S):N\cap N_G(S), |N\cap N_G(S)|)=1$. We now have two possibilites:
\begin{enumerate}
\item $N_G(S)=G\Rightarrow S\trianglelefteq G$.
\item We have a complement of $N\cap N_g(S)$ in $N_G(S)$ by using the fact that $G$ was a minimal counterexample.
\end{enumerate}
In the latter case, let $B$ be the complement, so that $|B|=|N_G(S):N\cap N_G(S)|=|G:N|$. Then $B$ is a complemet of $N$ in $G$, contradiction.

Now we claim that $N$ is abelian. By way of contradiction assume that $N$ is not abelian. Then $1<N'<N\trianglelefteq G$ and then $N'\trianglelefteq G$ because $N'$ is characteristic. The theorem holds for $N/N'\trianglelefteq G/N'$ because the counterexample was minimal. Therefore there exists $M/N'$ complement of $N/N'$ in $G/N'$. But 
\[
|N/N'|=\frac{|G/N'|}{|N/N'|}=|G/N|\text{ and } |M|=|M/N'||N'|=|G/N'||N'|<|G/N||N|=|G|
\]

We can apply the theorem to $M$ and $N'$ since $\gcd(|N'|,|G/N|)=1$, so there exists $B$ complement of $N'$ in $M$ with $|B|=|M|/|N'|=|G/N|$. So $B$ is a subgroup of $G$ and it is a complement of $N$ in $G$, a contradiction, so $N$ is abelian. The theorem then follows from the previous theorem.  

\end{dem}

\begin{teorema}[Hall]
Let $G$ be a finite solvable group and $\Pi$ a set of primes. Then
\begin{enumerate}
\item Hall $\Pi$-subgroups exist.
\item Two $\Pi$-subgroups are complement in $G$.
\item A $\Pi$-subgroup is contained in a Hall $\Pi$-subgroup. If $K$ is a $\Pi$-subgroup and $H$ is a Hall $\Pi$-subgroup, then there exists $g\in G$ such tat $K\leq H^g$. 
\end{enumerate}
\end{teorema}

\begin{teorema}[Hall]
If Hall $\Pi$-subgroups exist for all $\Pi$, then $G$ is solvable. 
\end{teorema}
\begin{dem}
Let $N$ a minimal normal subgroup of $G$. We can assume that $N<G$ and $N$ is elementary abelian for some prime $p$. We have to cases: $p\in \Pi$ and $p\notin\Pi$. Start with the first case. By induction, there exists $H/N$ a Hall $\Pi$-subgroup of $G/N$ and $|G:H|=|G/N:H/N|$, so $H$ is in fact a Hall $\Pi$-subgroup of $G$.

Let $H_1$ and $H_2$ two Hall-subgroups of $G$. Both contain $N$, so apply induction to $H_1/N$ and $H_2/N$ in $G/N$. We get that $\exists g\in G$ s.t. $(H_1/N)^g=(H_2/N)^g$, so $H_1^g=H_2^g$. Let $K$ be a $\Pi$-subgroup of $G$ in $H$ a Hall $\Pi$-subgroup of $G$. Then $KN$ is also a $\Pi$-subgroup of $G$. By induction, $KN/N$ is contained in a conjugate of $H$, so it is contained in a complement $H$. 

Now we prove the case $p\notin\Pi$. Remember that $N$ was a minimal normal subgroup $N<G$. We apply induction to $G/N$ and there exists $K/N$ a Hall $p$-subgroup of $G/N$. Since $p\notin\Pi$, $N\trianglelefteq K$, so by Zassenhaus $\exists H$ a Hall $\Pi$-subgroup of $K$ such that $|H|=|K|/|N|$, thus exactly the $\Pi$-subbroup of the order of $G$. We have $|H:N|=1$, so $H$ is a Hall 1-subgroup of $G$. $|H|=|K:N|=|G:H|=|G:K||N|$, each one is a $\Pi$-prime number

Now we want to prove that any two of them are conjugate. Let $H_1$ and $H_2$ be two Hall $\Pi$-subgroups of $G$, then $H_1N/N$ and $H_2N/N$ are Hall $p$-subgroups of $G/N$. By induction $\exists g\in G$ such that $H_1N=(H_2N)^g=H_2^gN$, byt $H_1$ is a Hall $\Pi$-subgroup of $H_1N$ and similarily with $H_2^g$. By Zassenhaus there exists $z\in H_1N=H_2^gN$ such that $H_1=(H_2^g)^z=H_1=H_2^{gz}$. 

Finally, let $K$ be a $\Pi$-subgroup of $G$ and let $H$ be a Hall $\Pi$-subgroup of $G$. We have that $KN/N$ is a $\Pi$-subgroup and $|KN/N|=|K/(N\cap K)|$. $HN/N$ is a $\Pi$-subgroup of $G$ and $|HN/N|=|H/(N\cap H)|=|H|$, so $KN/N$ is a contained in a conjugate of $HN/N$, meaning that $KN\leq (HN)^g=H^gN$ for some $g\in G$. We have $K\cap N=1$ since $|N|=p\notin\Pi$. $KN=KN\cap H^gN=(KN\cap H^g)N$ and $(KN\cap H^g)\cap N=1$, so we've found another complement of $N$ in $KN$. By Zassenhaus $\exists z\in KN$ s.t. $K=(KN\cap H^g)^z\leq (H^g)^z=H^{gz}$ and we are done.
\end{dem}

\begin{ej}
$|A_5|=\frac{5!}{2}=2^2\cdot 3\cdot 5$ and $A_5=\{1,(ab)(ca), (abc), (abcde)\}$. $A_4\subseteq A_5$ and $|A_4|=12$ and it has a complement: $\gene{12345)}$. If we think of a subgroup $G$ of order 15, it must be abelian, so if $G\cong C_3\times C_5$, $G$ should be generated by $(abc)\in S_3$ and $(a'b'c'd'd'e')$, but $(abc)^{(a'b'c'd'd'e')}\notin S_3$, so it must be cyclic. Since the sequence $1\to C_p\to C_{p^2}\to C_p\to 1$ does not split, $P\in Syl_2(A_5)$ has no complement in $A_5$. 
\end{ej}

\begin{prop}
Let $G$ be a finitely generated nilpotent group and suppose $G=\gene{Y}$. Then $\exists Z\subseteq Y$ finite with $G=\gene{Z}$.
\end{prop}
\begin{dem}
 $G^{ab}=G/G'$ is also finitely generated. $G^{ab}=\gene{\overline{y}\mid y\in Y}$. It is enought to do it for the abelian case. If $A_1=\gene{y_1}\subseteq A_2\gene{y_1,y_2}\subseteq\cdots$, since $\Z$-modules are noetherian (because $\Z$ is noetherian), this chain is stationary.
\end{dem}



\begin{lemma}
Let $G$ be a finite group and $H,K\leq G$. Suppose that $\gcd(|G:H|,|G:K|)=1$. Then $G=HK$. 
\end{lemma}
\begin{proof}
We have $HK\subseteq G$ and $|HK|=\frac{|H||K|}{|H\cap K|}$. We also have $|G:K|||G:H\cap K|$ and $|G:H|\mid|G:H\cap K|$ so using that $|G:K|$ and $|G:H|$ are coprime we get that $|G:K||G:H| \mid|G:H\cap K|$. Going to the firs equality we obtain 
\[
|HK|=\frac{|G:H\cap K||G|}{|G:H||G:K|}\geq |G|.
\]
\end{proof}

\begin{teorema}
Let $G$ be a finite group and $H_1,H_2,H_3\leq G$ solvable subgroups such that the $|G:H_i|$ are mutually coprime. Then $G$ is solvable.
\end{teorema}
\begin{dem}
By induction on $|G|$, there are two cases: either $H_1=1$ or not. In the first case, $G=H_2=H_3$ is solvable. In the second case, there exists a minimal normal subgroup $N\trianglelefteq H_1$. $N$ is elementary abelian for some prime $p$. Now either $p\nmid |G:H_2|$ or $p\nmid|G:H_3|$. We can assume $p\nmid|G:H_2|$. Let $D=H_1\cap H_2$. $ND\subseteq H_1\Rightarrow |ND:D|\mid|H_1:D|$, $|ND:D|=|N:N\cap D|\mid|N|=p^a$, $|ND:D|=1\Rightarrow N\leq D$. Take $M=\gene{N^g\mid g\in G}$ and take $g\in G$. Then $g=h_1h_2\Rightarrow N^g=N^{h_1h_2}=N^{h_2}\leq H_2$. Hence $M\leq H_2$ and $M$ is solvable ($M\neq 1$). We apply induction to $G/M$, $H_1M/M$ and $H_2M/M$ to get that $G/M$ is solvable, whence $G$.
\end{dem}

\begin{teorema}[Burnside]
Let $G$ be a finite group such that $|G|=p^aq^b$ Then $G$ is solvable.
\end{teorema}
The proof uses character theory.

\begin{ej}
Take $G$ as in the theorem and take Halls $\{p,q\},\{p,r\},\{q,r\}$ subgroups. By Burnside they are solvable, so $G$ is solvable by the previous theorem.
\end{ej}


\begin{teorema}[Hall]
Let $G$ be a finite group, $|G|=p_1^{a_1}\cdots p_r^{a_r}$ and assume that for all $i=1,\dots, n$, Hall $p_i$-subgroups exist. Then $G$ is solvable (all Syllow subgroups have complements).
\end{teorema}
\begin{dem}
By induction on $r$. For $r=1$, $G$ is a $p$-group, so $G$ is nilpotent and hence solvable. For $r=2$, $G$ is solvable by Burnside. Let $r\geq 3$ and $H_i$ a Hall $p_i$-subgroup with $|H_i|=|G|/p_i^{a_i}$. Let $B_{ij}=H_1\cap H_j\Rightarrow |B_{ij}|=|G|/p_i^{a_i}p_j^{a_j}$ and $|G:B_{ij}|=p_i^{a_i}p_j^{a_j}$. $B_{ij}$ is a Hall $\{p_i,p_j\}$-subgroup and $|H_i:B_{ij}|=p_j^{a_j}$ for all $j\neq i\Rightarrow H_i$ are solvable. By the previous theorem applied to any three of these subgroups, $G$ is solvable. 
\end{dem}
\begin{teorema}
Let $G$ be a finite group such that for all $\Pi$, Hall $\Pi$-subgroups exist. Then $G$ is solvable. 
\end{teorema}
\begin{ej}
$A_5$ does not contain a Hall $\{3,5\}$-subgroup.
\end{ej}



\section{Burnside problems and Engel groups}
In 1902, Burnside propose the \emph{Gen Burnside Problem}: let $G$ be a group. If $G$ is periodic ($\forall g\in G\exists m\mid g^m=1$) and finitely generated, is $G$ finite? There are cases when this is true. If $G$ is finite this is easy to prove. In 1911, Schur proved that every periodict subgroup of $GL_n(K)$ with a complex field ($n\geq 1$) is finite. This is also true for nilpotent groups and for solvable groups. In 1964, Safarevich found that for every $d>2$, there exists $G=\gene{x_1,\dots, x_d}$ such that \begin{enumerate}
\item $G$ is not nilpotent
\item $\forall H\leq G$ with $d(H)=d-1$ and $H$ is nilpotent
\item $G$ is a $p$-group, for $p$ prime
\item $G$ is infinite.
\end{enumerate}
In 1980 a new family of counterexamples was found, namely Grigorchuk groups. 

If a group is Engel and finitely generated, then $G$ is not necessarily finite but it is nilpotent. 

\begin{defi}
Let $G$ be a group. An element $g\in G$ is a \emph{left Engel} element if $\forall x\in G$ $\exists n=n(g,x)\geq 1$ sucht that $[x,_n g]=[x,g,\dots, g]=1$. The set of these elements is denoted by $L(G)$. Similarily we define a \emph{right Engel} element by $[g,_n x]=1$, denoted by $R(G)$. 
\end{defi}

We have that $R(G)^{-1}\subseteq L(G)$. The idea for this is using $[x,_{n+1}g]=[(g^{-1})^x,_ng]^g$. It is not known if $R(G)\subseteq L(G)$. The converse $R(G)\supseteq L(G)$ is not true in general. 

\begin{prop}
If $G$ is nilpotent, then it is Engel.
\end{prop}
\begin{dem}
Let $c\geq 1$ be the nilpotency class of $G$, $\gamma_{c+1}(G)=\{1\}$, so $[x,_cy]=1$ for all $x,y\in G$.

For all  $x,y\in G$, $\gene{x,y}$ is nilpotent, so for some $n$, $[x,_ny]=1$ for all $x,y\in G$.
\end{dem}

There are some cases in which the converse is also true. For example if $G$ is solvable or if it is finite. But in general this is not true, since Shafarevich group give counterexamples. 

\begin{teorema}[Zorn]
Every finite Engel group is nilpotent.
\end{teorema}
\begin{dem}
Induction on the order of $|G|$. if $|G|=1$ the theorem trivially follows. Assume $|G|=m>1$. Every subgroup $H<G$ is nilpotent. Assume by contradiction that $G$ is not nilpotent. Since $H$ is finite and Engel, $H$ is nilpotent. By Schmidt theorem, if $G$ is not nilpotent and every prober subgroup is nilpotent, then $G$ is solvable. This is the case here, but solvable groups are Engel, and $G$ is in adition finite, so it is nilpotent.
\end{dem}

Let us see the \emph{restricted} Burnside problem. If $G=\gene{x_1,\dots, x_m}$ is finite of $\exp(G)=n$, is there a bound for $|G|$ in terms of $n$ and $m$. The answer is yes.

Suppose that $G$ is a group generated by $d$ elements with $\exp(G)=3$. Then $G$ is finite and $|G|=3^{d+\binom{d}{2}+\binom{d}{3}}$.

\begin{prop}[Kappe and Kappe]
If $G$ has $\exp=3$, then $G$ is a 2-Engel group.
\end{prop} 
\begin{dem}
$xx^y=x^yx$ for all $x,y\in G$ and for all $g\in G$ $g^3=1$. $(xy^{-1})^3=1\Leftrightarrow (xy^{-1})^2=(xy^{-1})^{-1}\Leftrightarrow xy^{-1}xy^{-1}=yx^{-1}$. Multiplication on the right by $y^2$ gives us the result. 
\end{dem}

$G$ is 2-Engel iff $\gene{x}^G$ is abelian for all $x\in G$.


\section{Crown in solvable groups}
These wer introduced by Gaschütz in 1962 for solvable groups. The Ballester-Bollinches and Ezqueno defined them for general finite groups in 2006. 

\begin{defi}
Let $G$ be a finite group and $V$ a finite abelian group. We say that $V$ is a $G$-module if $G$ acts on $V$ and $(v+w)^g=v^g+w^g$ for every $v,w\in V$. A $G$-submodule of $V$ is a subgroup of $V$ which is a $G$-module. We say that $V$ is irreducible if it has no proper $G$-submodules. $\varphi:V\to W$ is a $G$-isomorphism if it an isomorphism of groups and $\varphi(v^g)=\varepsilon(v)^g$.
\end{defi}

\begin{ej}\
\begin{enumerate}
\item Every abelian group is a $\Z$-module.
\item If $M,N\trianglelefteq G$ such that $M/N$ is abelian, then $M/N$ is a $G$-module, where $G$ acts by conjugation. 
\end{enumerate}
\end{ej}

\begin{defi}
Let $G$ be a group. A series $1=N_n<N_{n-1}<\cdots<N_1<N_0=G$ is a \emph{chief series} if $N_i\trianglelefteq G$ and if $N_{i+1}\leq N\leq N_i$ and $N\trianglelefteq G$; then $N=N_i$ or $N=N_{i+1}$ for every $0\leq i\leq n-1$. We say that $N_i/N_{i+1}$ is a chief factor. We say that $N_i/N_{i+1}$ is a complemented chief factor if it is complemented in $G/N_{i+1}$.
\end{defi}

\begin{ej}\
\begin{enumerate}
\item
 If $G$ is a $p$-group, then $1=N_0\leq N_{n-1}\leq\cdots\leq N_0=G$ such that $N_i\trianglelefteq G$ and $N_i/N_{i+1}=C_p$.
 
 \item If $G$ is solvable and if $1=N_r<\cdots <N_0=G$ is a chief series, then $N_i/N_{i+1}$ is a minimal normal subgroup of $G/N_{i+1}$, so $N_i/N_{i+1}$ is abelian. In addition, there is no normal subgroup between $N_i$ and $N_{i+1}$, so $N_i/N_{i+1}$ is an irreducible $G$-module. 
\end{enumerate}
\end{ej}

\begin{teorema}[Jordan-Holdr] Let $G$ be a group and $1=N_n\leq\cdots\leq N_0=G$ and $1=M_m\leq\cdots\leq M_0=G$ two chief series. Then
\begin{enumerate}
\item if $n=m$ and $\exists\sigma\in S_n$ such that $N_i/N_{i+1}\cong M_{\sigma(i)}/M_{\sigma(i)+1}$
\item if $G$ is solvable, complemented chief factors correspond one to each other. 
\end{enumerate} 

\end{teorema}
From now on, $G$ is finite solvable, $V$ is an irreducible $G$-module.

\begin{defi}
Consider $\Delta(G,N)=\{N\trianglelefteq G\mid C_G(V)/N\cong_G V\}$. Let $R_G(V)=\bigcap_{N\in\Delta(G,V)}N$. Then $C_G(V)/R_G(V)$ is called the \emph{crown} of $G$ respect to $V$. 
\end{defi}


\begin{teorema}
$C_G(V)/R_G(V)\cong_G V\times\cdots\times V$ ($d_G(V)$ times). 
\end{teorema}

\begin{teorema}
All chief factor between $C_G(V)$ and $R_G(V)$ are complemented.
\end{teorema}

\begin{teorema}
There is no complemented chiaf factor $G$-isomorphic to $V$ above $C_G(V)$ or below $R_G(V)$. 
\end{teorema}

So actually, $d_G(V)$ is precisely the number of complemented chief factors $G$-isomorphic to $V$ of any chief series.

\begin{teorema}
If $\Phi(G)=1$, then $\exists D\cong V\overset{d_G(V)}{\times\cdots\times}V$ such that $C_G(V)=R_G(V)\times D$. 
\end{teorema}

\end{document}